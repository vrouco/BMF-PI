\documentclass[,man,floatsintext]{apa6}
\usepackage{lmodern}
\usepackage{amssymb,amsmath}
\usepackage{ifxetex,ifluatex}
\usepackage{fixltx2e} % provides \textsubscript
\ifnum 0\ifxetex 1\fi\ifluatex 1\fi=0 % if pdftex
  \usepackage[T1]{fontenc}
  \usepackage[utf8]{inputenc}
\else % if luatex or xelatex
  \ifxetex
    \usepackage{mathspec}
  \else
    \usepackage{fontspec}
  \fi
  \defaultfontfeatures{Ligatures=TeX,Scale=MatchLowercase}
\fi
% use upquote if available, for straight quotes in verbatim environments
\IfFileExists{upquote.sty}{\usepackage{upquote}}{}
% use microtype if available
\IfFileExists{microtype.sty}{%
\usepackage{microtype}
\UseMicrotypeSet[protrusion]{basicmath} % disable protrusion for tt fonts
}{}
\usepackage{hyperref}
\hypersetup{unicode=true,
            pdftitle={The Berliner Multi-Facet Personality Inventory: An extensive measure of Big Five personality},
            pdfauthor={Victor Rouco, Anja Cengia, \& Matthias Ziegler},
            pdfkeywords={keywords},
            pdfborder={0 0 0},
            breaklinks=true}
\urlstyle{same}  % don't use monospace font for urls
\usepackage{graphicx,grffile}
\makeatletter
\def\maxwidth{\ifdim\Gin@nat@width>\linewidth\linewidth\else\Gin@nat@width\fi}
\def\maxheight{\ifdim\Gin@nat@height>\textheight\textheight\else\Gin@nat@height\fi}
\makeatother
% Scale images if necessary, so that they will not overflow the page
% margins by default, and it is still possible to overwrite the defaults
% using explicit options in \includegraphics[width, height, ...]{}
\setkeys{Gin}{width=\maxwidth,height=\maxheight,keepaspectratio}
\IfFileExists{parskip.sty}{%
\usepackage{parskip}
}{% else
\setlength{\parindent}{0pt}
\setlength{\parskip}{6pt plus 2pt minus 1pt}
}
\setlength{\emergencystretch}{3em}  % prevent overfull lines
\providecommand{\tightlist}{%
  \setlength{\itemsep}{0pt}\setlength{\parskip}{0pt}}
\setcounter{secnumdepth}{0}
% Redefines (sub)paragraphs to behave more like sections
\ifx\paragraph\undefined\else
\let\oldparagraph\paragraph
\renewcommand{\paragraph}[1]{\oldparagraph{#1}\mbox{}}
\fi
\ifx\subparagraph\undefined\else
\let\oldsubparagraph\subparagraph
\renewcommand{\subparagraph}[1]{\oldsubparagraph{#1}\mbox{}}
\fi

%%% Use protect on footnotes to avoid problems with footnotes in titles
\let\rmarkdownfootnote\footnote%
\def\footnote{\protect\rmarkdownfootnote}


  \title{The Berliner Multi-Facet Personality Inventory: An extensive measure of
Big Five personality}
    \author{Victor Rouco\textsuperscript{1,2}, Anja Cengia\textsuperscript{3}, \&
Matthias Ziegler\textsuperscript{3}}
    \date{}
  
\shorttitle{BMF-PI}
\affiliation{
\vspace{0.5cm}
\textsuperscript{1} Universitat de Barcelona\\\textsuperscript{2} Institut de Neurociències de Barcelona\\\textsuperscript{3} Humboldt Universität zu Berlin}
\keywords{keywords\newline\indent Word count: X}
\usepackage{csquotes}
\usepackage{upgreek}
\captionsetup{font=singlespacing,justification=justified}

\usepackage{longtable}
\usepackage{lscape}
\usepackage{multirow}
\usepackage{tabularx}
\usepackage[flushleft]{threeparttable}
\usepackage{threeparttablex}

\newenvironment{lltable}{\begin{landscape}\begin{center}\begin{ThreePartTable}}{\end{ThreePartTable}\end{center}\end{landscape}}

\makeatletter
\newcommand\LastLTentrywidth{1em}
\newlength\longtablewidth
\setlength{\longtablewidth}{1in}
\newcommand{\getlongtablewidth}{\begingroup \ifcsname LT@\roman{LT@tables}\endcsname \global\longtablewidth=0pt \renewcommand{\LT@entry}[2]{\global\advance\longtablewidth by ##2\relax\gdef\LastLTentrywidth{##2}}\@nameuse{LT@\roman{LT@tables}} \fi \endgroup}


\usepackage{lineno}

\linenumbers
\usepackage[titles]{tocloft}
\cftpagenumbersoff{table}
\renewcommand{\cfttabpresnum}{\itshape\tablename\enspace}
\renewcommand{\cfttabaftersnum}{.\space}
\setlength{\cfttabindent}{0pt}
\setlength{\cftafterloftitleskip}{0pt}
\settowidth{\cfttabnumwidth}{Table 10.\qquad}
\usepackage{rotating}
\usepackage{longtable}
\usepackage{changepage}

\authornote{Add complete departmental affiliations for each
author here. Each new line herein must be indented, like this line.

Enter author note here.

Correspondence concerning this article should be addressed to Victor
Rouco, Postal address. E-mail:
\href{mailto:victorrouco@ub.edu}{\nolinkurl{victorrouco@ub.edu}}}

\abstract{
Enter abstract here. Each new line herein must be indented, like this
line.


}

\begin{document}
\maketitle

\section{1. Introduction}\label{introduction}

Over the last decades, the Five Factor Model as well as the Big Five
model have become widely accepted models for describing general
attributes of personality. Often the terms are even used synonymously,
which is why we will refer to the Big Five from here on. The Big Five is
a hierarchical model which describes human individual differences in
personality at the dispositional level: one of the most basic,
universal, biologically-influenced and stable layers of human
inter-individual differences in behavior, cognition and feeling (D. P.
McAdams \& Pals, 2006). Its hierarchical conception is relevant to
acknowledge behavior from the most specific (nuances) to the most broad
differences in temperament and character (dimensions), through a varying
number of mid-level personality characteristics (facets). Most of the
research concerning criterion validity of the Big Five inventories has
focused on the covariation between the Big Five dimensions and relevant
external outcomes. However, specific dispositional characteristics
captured on the facet level might be of extreme utility to provide more
complex descriptions of individuality and to predict life outcomes to a
major extent (Lounsbury, Sundstrom, Loveland, \& Gibson, 2002; S. V.
Paunonen \& Ashton, 2001; Ziegler et al., 2014). Unfortunately, the
number and nature of the facets below the Big Five is far from being
consensual. In fact, different facet level models have been proposed
(XXXX). One potential reason for this could be that many facet level
models were developed after a questionnaire version without such a level
had been published. Thus, the facets were developed as an elaboration or
extension to an existing domain measure. While this has many theoretical
advantages it also has the disadvantage of potentially limiting the
search space of possible facets. In this work we aim at maximizing this
search space and present a personality questionnaire which is broad at
the facet level, open-access, and measurement invariant across two
different cultures.

\subsection{1.2. A short history of the Big
Five}\label{a-short-history-of-the-big-five}

Francis Galton is credited as being the one who proposed the fundamental
lexical hypothesis as a ground from where to describe interpersonal
differences in personality. The hypothesis states that every apprehended
characteristic in the realm of personality should have its place in the
natural language, a corollary derived from this first statement is that
the essential features must represent a unique word in the lexical
universe of this language. Galton (1884) himself, and later Allport and
Odbert (1936) and still later Norman (1967), used English dictionaries
for a systematic collection of all adjectives which could be related to
human personality characteristics. Using exploratory factor analyses on
self and other ratings, five broad factors could repeatedly be extracted
from the data.

Cattell was one of the first researchers who systematically applied
exploratory factor analysis in order to explore personality structure.
He inspected the correlation structure of the items in the word lists of
his predecessors, finding 16 oblique personality factors, including one
factor specifically for intelligence, these factors form the 16-PF.
These 16 factors were the primary factors in a hierarchical structure
for Cattell (coetany to L.L. Thurstone and undoubtedly influenced by
him). Cattell himself viewed personality as a hierarchical structure,
containing three layers (Cattell, 1956). The second order factors
resemble the Big Five dimensions (Digman, 1990).

Different researchers followed Cattell in the study of dispositional
traits of personality. One of the most influential models was Eysenck`s
Big Three. Grounded on a strong biological basis, Eysenck`s theory
supposed a link between temperament and personality. Its structural
proposal concerned at first two big factors, named Neuroticism
vs.~Emotional stability and Extraversion vs.~Introversion. These two
dimensions were later joined by a third factor that Eysenck called
Psychoticism. This label was criticized by others who suggested that a
more appropriate term would be psychopathy (Digman, 1990). Eysenck`s big
two are still \enquote{alive}" today in the Big Five, and his third
factor, psychoticism, can be operationalized as other dimensions within
the Big Five: Agreeableness, Conscientiousness and Openness.

A large number of studies have focused on the problem of personality
structure resulting in a five factor solution (Borgatta, 1964; Fiske,
1949; Norman, 1967; Tupes \& Christal, 1961). Possibly the two most
widely cited works relating to the foundations of the Big Five are those
by Goldberg et al. (2006) and P. T. Costa and McCrae (1995). Goldberg
can be seen as one of the first who extended research concerning the Big
Five, while McRae and Costa`s importance rests on popularizing the
terminology (OCEAN) and the development of one of the most used tools to
assess personality based on the Big Five: the NEO-PI. The Big Five
dimensions are labeled as follows: I) Extraversion vs.~Introversion. II)
Agreeableness or Friendliness. III) Conscientiousness or Achievement or
Will. IV) Emotional Stability vs.~Neuroticism, and V) Openness or
Intellect or Culture.

One of the most important features of the Big Five is the fact that it
could be replicated in different languages. Research is available in
Japanese, Vietnamese, German, Spanish, Greek, and many more (Schmitt et
al., 2007). This finding suggests that the way human beings construe
personality is at some point universal and that its basic features are
retained within the Big Five. Another essential characteristic relies on
its hierarchical nature. The five domains are useful to retain the big
picture of personality, maximize the situation consistency and reliably
assess difficult subjects such as children. Nonetheless, each dimension
is conceptualized as a latent construct formed by more specific narrow
factors called facets, which in turn are useful to depict the impact of
personality characteristics into specific behaviors and concrete life
outcomes.

The Big Five has proven to be a valid theoretical and empirical model to
predict relevant life outcomes. Research such as Ozer and Benet-Martínez
(2006) or B. W. Roberts, Kuncel, Shiner, Caspi, and Goldberg (2007) has
shown that scores for the Big Five dimensions (and their related facets)
are able to explain outcomes such as academic and work performance,
health, personality disorders, political attitudes and many more. The
empirical findings linking Big Five measures to life outcomes have
reinforced the concurrent validity of the test scores interpretations.
At the same time, the broad nature of the domains has spurned research
into the more fine-grained lower order structure of facets.

\subsection{1.3. Facet Structures}\label{facet-structures}

There are a number of models that include a facet structure below the
five broad domains. The most widely known model is the one suggested by
P. T. Costa and McCrae (1995), the NEO-PI-R model. Other popular models
have been suggested for the Big Five Inventory 2 (BFI-2; Christopher J
Soto \& John, 2016), the IPIP (Goldberg et al., 2006), and the HEXACO
model (K. Lee \& Ashton, 2016), which assumes six broad domains.
\emph{Table 1} gives an overview of these different models listing their
facets per domain as well as some information regarding their
psychometric properties.

\vspace{5mm}

\textless{} Table 1 \textgreater{}

\vspace{5mm}

As shown in \emph{Table 1}, there are different possibilities of facets
forming the domains. However, there is still a degree of overlap between
the facets covered by the different instruments. Christopher J. Soto and
John (2009) inspected the convergences between the NEO-PI-R and the
first version of the BFI, suggesting that two constructs per domain were
measured at the facet level by both inventories. The constructs defined
by Christopher J. Soto and John (2009) were: \emph{Altruism} and
\emph{Compliance} for Agreeableness; \emph{Anxiety} and
\emph{Depression} for Neuroticism; \emph{Order} and
\emph{Self-Discipline} for Conscientiousness; \emph{Assertiveness} and
\emph{Activity} for Extraversion; and \emph{Aesthetics} and \emph{Ideas}
for Openness. The convergence holds for the four instruments listed in
\emph{Table 1}, as these ten constructs are covered within the facets
for every instrument. Some of the constructs are explicitly covered at
the facet level (e.g.~Anxiety); meanwhile others are mainly covered by
the four instruments, although sometimes implicitly (e.g.~Liveliness in
HEXACO resembles the \enquote{core} construct Activity, present in all
other instruments). The reverse is not always true, not every facet
within the four instruments is covered by the constructs proposed by
Christopher J. Soto and John (2009). As an example we find
Self-Consciousness, a Neuroticism facet defined by the NEO-PI-R and the
IPIP-NEO-120, which is not intrinsically tapping at either Anxiety or
Depression. The same authors asserted in a later work (Christopher J
Soto \& John, 2016) that the Big Five domains \emph{\enquote{can be
conceptualized and assessed more broadly or more narrowly}}, either
focusing in a central facet or in a set of peripheral facets, depending
the research interest.

The mid-level layer between domains and facets has also been explored by
DeYoung, Quilty, and Peterson (2007). Their work has focused in the
biological consistency of the NEO-PI-R set of facets, thereby proposing
a two factor source of variance for each facet of the inventory. In line
with their proposal, Agreeableness would be composed by
\emph{Compassion} and \emph{Politeness}; Neuroticism by
\emph{Volatility} and \emph{Withdrawal}; Conscientiousness by
\emph{Industriousness} and \emph{Orderliness}; Extraversion by
\emph{Enthusiasm} and \emph{Assertiveness}; and Openness by
\emph{Intellect} and \emph{Openness}. Both Christopher J. Soto and John
(2009) and DeYoung et al. (2007) proposals have many points in common.
Maybe the labels \emph{Volatility} and \emph{Withdrawal} for Neuroticism
can be suspicious of a different content than \emph{Anxiety} and
\emph{Depression}, but when inspected at the item level it is revealed
that they are tapping the same components respectively (DeYoung et al.
(2007); for item specification).

The nomological network commonly assumed in Big Five questionnaires is
drawn from nuances through facets to domains, from more specific to more
general. Relying on domains to explain and predict behavior can benefit
from ease of interpretability. However, predictions for specific
contexts can be enhanced if a more specific set of traits is used. On
the other hand, using nuances to predict behavior might yield even
stronger predictive ability (Seeboth \& Mõttus, 2018), but as the number
of predictors grows the interpretations become more complex. Facets are
on a middle ground between nuances and domains, in a compromise between
specificity and sensitivity in the bandwidth-fidelity dilemma. This
narrow aggregation both satisfies the specificity of predictions to
concrete situations and environments and also enhances the ease of
interpretability when summarizing individual personality
characteristics.

Personality measured at the facet level has found to be a strong
predictor of a large number of outcomes. Satisfaction with life (SWL) is
one of them. Neuroticism and Extraversion were recognised as the most
important personality dimensions in the prediction of subjective
satisfaction (Diener, Oishi, \& Lucas, 2003; Schimmack, Diener, \&
Oishi, 2002). Lately, Schimmack, Oishi, Furr, and Funder (2004) observed
that the analysis at the facet level outperform the analysis at the
domain level. They observed that \emph{Depression} and \emph{Positive
Emotions} / \emph{Cheerfulness} explained SWL above and beyond the
dimensions they belong to, reaching to a 30\% of explained variability
of SWL. Correlations in the Schimmack et al. (2004) study ranged in a
longitudinal design from \emph{r} = -.57 to \emph{r} = -.49 for the
first and from \emph{r} = .51 to \emph{r} = .38 for the second and
third.

Another relevant outcome that has shown to be best predicted with
personality at the facet level is academic achievement. The relation of
Conscientiousness with academic performance has gained a stable
empirical evidence, with correlations ranging from \emph{r} = .20 to
\emph{r} = .45 depending in sample specifity (Chamorro-Premuzic \&
Furnham, 2003; De Fruyt \& Mervielde, 1996; Lievens, Coetsier, De Fruyt,
\& De Maeseneer, 2002; Noftle \& Robins, 2007; O'Connor \& Paunonen,
2007; S. V. Paunonen \& Ashton, 2001; Poropat, 2009, 2014; D. Watson \&
Watson, 2002). De Fruyt and Mervielde (1996) hypothesized that
volitional facets of Conscientiousness would be more proned to exhibit
strong relations with academic achievement. In this line, there is a
collection of research which points at relations of GPA scores with
facets such as \emph{Achievement-striving} (Chamorro-Premuzic \&
Furnham, 2003; O'Connor \& Paunonen, 2007 \emph{r} ranging from .15 to
.39; D. Watson \& Watson, 2002 \emph{r} = .39) or \emph{Work drive}
(Lounsbury et al., 2002, \emph{r} = .12). Nonetheless, also other
Conscientiousness facets more related to duties or moral driveness have
been found to predict significantly GPA scores, like
\emph{Self-discipline} (O'Connor \& Paunonen, 2007, \emph{r} ranging
from .18 to .25; D. Watson \& Watson, 2002, \emph{r} = .36) or
\emph{Dutifulness} (Chamorro-Premuzic \& Furnham, 2003; O'Connor \&
Paunonen, 2007, \emph{r} ranging from .25 to .38). The relation of
academic achievement with Openness has been more variant. Following the
categories proposed by Costa \& McCrae, students which showed both high
Conscientiousness and high Openness would be considered \enquote{good
students}. Moreover, those who score high in Openness but not in
Conscientiousness were labelled \enquote{dreamers} and their performance
in academic test is less stable. Some studies found a significant
relation between the Openness dimension and academic achievement
(Lievens et al., 2002, \emph{r} = .09; D. Watson \& Watson, 2002,
\emph{r} = .18), while some others failed to replicate this findings
(Chamorro-Premuzic \& Furnham, 2003; S. V. Paunonen \& Ashton, 2001,
\emph{r} = -.04). Is within this dimension were facet level analysis may
be hugely useful. S. V. Paunonen and Ashton (2001) found that the
Openness facet of \emph{Understanding} correlates with academic
achievement with a \emph{r} = .23. Noftle and Robins (2007) identified a
set of NEO-PI-R and HEXACO's Openness facets which predicted academic
achievement (the HEXACO facets of \emph{Aesthetic},
\emph{Inquisitiveness}, \emph{Creativity} and \emph{Unconventionality},
plus the NEO-PI-R facets of \emph{Fantasy}, \emph{Aesthetics},
\emph{Feelings} and \emph{Ideas}). Oppositely, Ziegler et al. (2014)
found that \emph{Openness to ideas} was related positively with work
performance, while \emph{Openness to fantasy} was related negatively,
potentially masking the overall effect of Opennes over the working
performance criterion. In this study we aim to get deep into the
research question of which facets are involved in scholastic
achievement, using a widely facetted inventory. Moreover, narrow level
analysis seems to improve the predicitive power of personality on
academic performance, adding about 10\% of explained variance
(Lounsbury, Steel, Loveland, \& Gibson, 2004; O'Connor \& Paunonen,
2007; Ziegler, Danay, Schölmerich, \& Bühner, 2010).

Likewise, personality has proven to be a powerful predictor of laboral
and educational abseentism (Chamorro-Premuzic \& Furnham, 2003; Judge,
Martocchio, \& Thoresen, 1997; Salgado, 2002). Research has highlighted
the predictive power of personality test over the so-called integrity
test when predicting absences (Ones, Viswesvaran, \& Schmidt, 2003).
Again, most research has focused on the dimensional level, although some
researchers suggested that personality assessed at a narrower level
would improve the predictive ability of the models (Lounsbury et al.,
2004; Salgado, 2002). Nonetheless, few studies have explored this
relationship to our knowledge, being Lounsbury et al. (2004) and Judge
et al. (1997) the most prominent. Judge et al. (1997) dreported no
predictive gain when examining personality at the facet level for the
NEO-PI-R composites of Extraversion and Conscientiousness, whereas
Lounsbury et al. (2004) found a modest predictive gain of \emph{Work
drive} over the Big Five dimensions. Therefore, and despite the
conceptual expectation of facets maximizing the predictive ability of
personality on abseentism, evidence has manifested in favour of a
dimension level analysis. However, it can be arguable that more research
needs to be done in this area, preferably using personality inventories
which are broad at the facet level.

As described above, facet measures often yield scores that have stronger
test-criterion correlations than their respective domain scores.
However, facet scores have also been shown to be related to personality
disorders. Thus, the combination of a higher fidelity along with the
potential clinical relevance of facet scores might open up unique
advantages for clinical research.

\subsection{1.4. The Big Five and Personality
Disorders}\label{the-big-five-and-personality-disorders}

Personality disorders are steadily shifting from a categorical
definition into a continua conceptualization within the clinical realm.
This process is not new for personality science history, as the subject
itself moved from a qualitatively distinct set of definitions, called
types, into a subset of continuous domains in which both normality and
extreme tendencies were moving along, named traits. In fact, the new
version of the Diagnostic and Statistical Manual of mental disorders,
DSM-V, now proposes two different ways of assessing personality
disorders: 1) A descriptive model of personality disorders in section II
which mimics the former model of assessing personality disorders and; 2)
A novel trait model that follows research on the personality scientific
domain (In section III), which conceptualizes personality disorders as
extreme tendencies located in the continuum of the Big Five domains and
facets (American Psychiatric Association, 2013; T. A. Widiger \&
Mullins-Sweatt, 2009)

This paradigm shift in clinical assessment of personality has led to the
construction of the Personality Disorder Inventory (PID-5; R. F.
Krueger, Derringer, Markon, Watson, \& Skodol, 2012), a 25-facet and
five-dimension self-report inventory, with an informant-report version
(K. E. Markon, Quilty, Bagby, \& Krueger, 2013). These five dimensions
mirror the Big Five domains, although with a focus on the maladaptative
end of the continuum,: I) Detachment (Big Five's Introversion), II)
Antagonism (absence of Big Five's Agreeableness), III) Disinhibition
(absence of Big Five's Conscientiousness), IV) Negative affect (Big
Five's Neuroticism) and V) Psychoticism (Absence of Big Five's
Openness). The PID-5 has shown satisfying evidences of criterion
validity (\ldots{}summary). However, the number of facets per domain on
the PID-5 is limited.

In line with what has been stated previously for academic achievement,
the examination of facets may result in an enhancement of the
specificity of assessment when looking at the nature of PDs (L. A.
Clark, 2005; Samuel \& Widiger, 2008). This improvement of specificity
resulted in a predictive gain ranging from 3\% to 16\% when comparing
facets to domains predicting PD in a study by Reynolds and Clark (2001).
Furthermore, the use of facets may be of extreme utility for those PD
whose personality profile is less clear at the domain level. As Saulsman
and Page (2004) pointed out, Schizotypal and Obsessive-Compulsive
disorders are examples of PD which are not well covered by Big Five
domains. A reason for it may be found in a pattern inconsistency of
facets within the same dimension or in a lack of coverage for essential
characteristics of the PD. For example, aberrant cognitions are
essential characteristics of schizotypal disorder and are not covered by
some instrument's facets like the NEO-PI-R (Samuel \& Widiger, 2008;
Saulsman \& Page, 2004). Likewise, the expected high scores on warmth
and low scores on assertiveness could mask the effects of extraversion
when predicting Dependent Personality Disorder, following the
theoretical correspondence between PD and Big Five facets proposed by
Costa Jr. and Widiger (1994). Moreover, the PID-5 has prompted the
elaboration of a number of Five Factor Model Personality Disorders
(FFMPD) scales to maximize the facet coverage in relation to specific
PDs (R. M. Bagby \& Widiger, 2018).

Facet analysis and dedicated Big Five questionnaires have been used to
solve issues like those mentioned in the last paragraph. We propose to
base such research on a broader facet basis. To this end we suggest a
general instrument to cover a broad number of facets which could aim for
fine grained assessments.

\subsection{This study}\label{this-study}

We present in this paper an instrument for personality assessment which
aims to cover the need for an internationally usable, open source, and
differentiated measure at the facet level. Two studies are presented,
for each one inspects the factor structure of the instrument in a
different sample drawn from a different culture (American vs.~German).
In the first study we develop the instrument by confirming a factorial
structure found after fitting an exploratory factor analysis.
Reliability indices are provided for the facets. Furthermore, we use the
found facets to predict external outcomes and thereby provide evidences
of criterion validity. We aim to test the following hypothesis, designed
to replicate previous findings:

\begin{itemize}
\tightlist
\item
  H1. SWL will be best predicted by the composites of Extraversion and
  Neuroticism.

  \begin{itemize}
  \tightlist
  \item
    H1.1. Adding the facets will significantly improve the predictions
    of personality on SWL.
  \item
    H1.2. Particularly the facets \emph{Confidence} (N2) and
    \emph{Positive attitude} (E4) will behave similarly to those
    reported by Schimmack et al. (2004).
  \end{itemize}
\item
  H2. Conscientiousness will be the strongest dimension when predicting
  academic achievement.

  \begin{itemize}
  \tightlist
  \item
    H2.1. Openness will be related positively but moderately to academic
    achievement.
  \item
    H2.2. Facets will add about 10\% of additional explained variance to
    dimensions when predicting academic achievement
  \end{itemize}
\item
  H3. Facets will improve the predictive power of dimensions when
  predicting school abseentism.
\end{itemize}

Furthermore, we aim to provide evidences on the research questions of
which facets predominantly correlate with academic achievement and
school abseentism. Measurement invariance across samples will be
examined in the second study. To sum up, the aim for this research
project was to provide an instrument that can be used in non-clinical
but also in clinical research which emphasizes the facet level of the
Big Five.

\section{Methods}\label{methods}

Two different studies are presented in this work. The first study uses a
sample drawn from the USA bachelor student population. The aim was to
detect and confirm a model that maximizes the facet space below the Big
Five domains. Exploratory factor analysis (EFA) was used to identify the
number of facets per domain. A confirmatory factor analysis (CFA) per
facet was specified in order to confirm the item - facet relationship.
An exploratory strucutural equation model (ESEM) was applied to test a
full model in which the facets serve as indicators of the Big Five
domains. ESEM has gained reputation in the personality field, where the
independent cluster model may not capture the complexity of the
constructs measured (Marsh et al., 2010). Finally, reliability measures
for the facets and test-criterion correlations will be computed to
achieve evidences of reliability and criterion validity.

The second sample is a sample representative for the German speaking
population of Germany, Austria and Switzerland. The aim for the second
study was to replicate the structure found in study one, plus assess the
degree of measurement invariance of the proposed model.

\subsection{Study 1 - US-American
Sample}\label{study-1---us-american-sample}

\subsubsection{Participants}\label{participants}

The sample consisted of 722 American undergraduate students (59.30\%
male) who participated voluntarily. The mean age was 21.60 years (SD =
5.90). Students were emailed a link to a computerized assessment battery
that included the IPIP items as well as several other tests not reported
in this paper. The data set was randomly split into two equally sized
samples. Both samples were matched in relation to missing values,
outliers and extreme values. In Sample 1 the mean age was 21.80 years
(SD= 6.30), in Sample 2 the mean age was 21.50 years (SD=5.60).

\subsection{Measures}\label{measures}

\subsubsection{Items from the International Personality Item Pool
(IPIP)}\label{items-from-the-international-personality-item-pool-ipip}

Altogether, 525 items from the \emph{International Personality Item
Pool} (IPIP) were used to measure Neuroticism, Extraversion, Openness
(to experience), Agreeableness and Conscientiousness. The IPIP is an
open source database of personality items, which was launched in 1996,
and contains over 2000 items (Goldberg et al., 2006). Participants were
asked to rate themselves on typical behaviors or reactions on a 5-point
Likert scale, ranging from 1 (\enquote{Not all like me}) to 5
(\enquote{Very much like me}).

The item selection was part of a different project and the procedure has
been explained in detail in the appendix of a study by MacCann,
Duckworth, and Roberts (2009). That study also contains part of the
sample used here. However, the current data set contains more
participants.

\subsubsection{Satisfaction With Life
(SWL)}\label{satisfaction-with-life-swl}

Measured with a 5 item composite defined in Diener, Emmons, Larsen, and
Griffin (1985), answered in a 7 point Likert-type scale ranging from 1
(\emph{strongly disagree}) to 7 (\emph{strongly agree}). The items are:
a)\enquote{In most ways my life is close to ideal}, b) \enquote{The
conditions of my life are excellent}, c) \enquote{I am satisfied with my
life}, d) \enquote{So far I have gotten the important things in my
life}, and e) \enquote{If I could live my life over, I would change
almost nothing}. Psychometric properties have been reported excellent.
(In which moment have SWLS been measured? just at the same time point
than personality?)

\subsubsection{GPA}\label{gpa}

Grade Point Averages measured in high school, university, and in cc.

\subsubsection{SAT}\label{sat}

\subsubsection{ACT}\label{act}

\subsubsection{Absences from class}\label{absences-from-class}

As a behavioral measure absence from class was asked to report from
subjects. ABS2 ABS4 what are the differences?

\subsection{2.1.3. Procedure}\label{procedure}

\subsubsection{EFA with subsample 1}\label{efa-with-subsample-1}

To determine the number of possible facets per domain Velicer (1976)
Minimum Average Partial (MAP) method and Horn (1965) parallel analysis
(PA) were employed for every domain. Based on these results an
Exploratory Factor Analysis (EFA) was calculated for each domain via
Mplus using a geomin rotation (Quelle) and a Maximum Likelihood
estimator (ML). The decision for the preferred number of facets per
domain was based partly on comparing model fits (CFI, RMSEA, SRMR). More
importantly though was the interpretability of the facet solution. To
this end, facet solutions from other personality measures were looked
and compared to the found facet structure. If there were important parts
missing to present the domain with regards to content, new facets were
added a posteriori.

\subsubsection{CFA and ESEM with subsample
1}\label{cfa-and-esem-with-subsample-1}

To confirm the structure of facets the EFAs delivered, multiple CFAs
were calculated via Mplus. In a first step, measurement models were
estimated for each of the facets. To obtain balance between the facets,
the items were reduced to five per facet based on item content and
loading pattern. In a second step, the estimations for the measurement
models on facet levels were repeated via CFA. For both steps estimators
were WLSMV (Weighted Least Squares adjusted for Means and Variances).
Aim was to ensure an optimal breadth and sufficient reliability. In a
final model, all five domain structural models were integrated using
ESEM (Asparouhov \& Muthén, 2009). Marsh et al. (2010) could show that
ESEM fits personality data better and results in substantially more
differentiated factors than CFA. All facets were allowed to load on all
domains. If there would show up facets that do not significantly load on
the intended domain, this facets would get eliminated subsequently. The
estimator used was WLSMV, factor scores from the facet CFAs were used as
indicators and the rotation was oblique (using Geomin). Model fit was
determined based on the guide lines by Hu and Bentler (1999) as well as
Beauducel and Wittmann (2005). Consequently, to consider a good fit of a
proposed model, the Comparative Fit Index (CFI) should be at or over
.95, the Standardized Root Mean Squared Residual (SRMR) smaller than .08
and the Root Mean Square Error of Approximation (RMSEA) smaller than
.06. For the ESEM models we compared our results with the findings by
Marsh et al. (2010).

\subsection{Reliability}\label{reliability}

Chronbach's \(\alpha\) and McDonald's \(\omega\) will be calculated for
each facet to provide measures of internal consistency.

\subsubsection{Criterion validity
evidence}\label{criterion-validity-evidence}

To examine the nomological structure of the facets and domains to
external constructs, a set of linear models and correlations were
fitted. We describe in this section the methods used to test the set of
hypothesis described in the introduction.

To explore H1 we explored the correlation matrix between Big Five
dimensions and SWL. To explore H1.1. we fitted a two-step regression
including the full set of dimensions in the first step and a selection
of facets following a stepwise procedure in the second step. H1.2. will
be inspected adding only N2 and E4 to the second step instead of the
full set of facets.

H2 and H2.1. will be tested by examining the correlation matrix of
Conscientiousness and Openness with academic achievement. A stepwise
regression will be used to test H2.2. Conscientiousness and Openness
will be the first set of predictors and their respective facets will be
entered in a second step, changes in \(R^2\) will be inspected.

H3 will also be tested with a hierarchical regression in which the Big
Five dimensions will be first entered and then a set of facets
previously selected by stepwise regression from the full set.

The research questions will be commented by looking at which facets best
predict academic achievement and scholastic absences.

\subsection{Results}\label{results}

\subsubsection{Results of EFA}\label{results-of-efa}

In \emph{Table 2} model fits for the chosen facet model for each domain
are shown, as well as Eigenvalues and results from MAP and PA test. To
ensure the homogeneity of the facets and to reduce the risk of cross
domain loadings, items with factor loadings less than .30 were
eliminated. This was only done when item content was also judged as
being non-central to the domain in question (Ziegler et al., 2014).

\vspace{5mm}

\textless{} Table 2 here\textgreater{}

\vspace{5mm}

According to the exploratory model, Agreeableness consists of eight
facets after two facets were eliminated due to weakly loading and
inconsistent items. The remaining facets were named \emph{Appreciation},
\emph{Integrity}, \emph{Low competitiveness}, \emph{Readiness to give
feedback}, \emph{Search for support}, \emph{Good faith},
\emph{Genuineness} and \emph{Altruism}.

Conscientiousness consists of nine facets after one facet with item
factor loadings less than .30 was excluded, they are: \emph{Dominance},
\emph{Persistence}, \emph{Self-discipline}, \emph{Task planning},
\emph{Goal orientation}, \emph{Carefulness}, \emph{Orderliness},
\emph{Wish to work} (to capacity) and \emph{Productivity}.

Extraversion is formed by nine facets. A new facet (\emph{Energy}) was
added in order to tap better the physical component of Extraversion,
which was missing in the eight facet solution. The facets are
\emph{Sociability}, \emph{Readiness to take risks}, \emph{Wish for
affiliation}, \emph{Positive attitude}, \emph{Forcefulness},
\emph{Communicativeness}, \emph{Humor}, \emph{Conviviality} and
\emph{Energy}.

Neuroticism (interpreted here as emotional stability) consists of seven
facets. One facet was dropped due to poor interpretability, and was
therefore not included in the subsequent analyses. The final set of
facets are named \emph{Equanimity}, \emph{Confidence},
\emph{Carefreeness}, \emph{Mental balance}, \emph{Drive},
\emph{Emotional robustness} and \emph{Self-attention}.

Openness to experience comprises nine facets. One facet was identified
as a method factor and eliminated, because it solely contained
negatively formulated items and no coherent underlying trait could be
identified. Furthermore another facet (\emph{Intellect}) was added,
because the remaining facets lacked an intellectual content. The facets
of Openness are named \emph{Creativity}, \emph{Wish for variety},
\emph{Open-mindedness}, \emph{Interest in reading}, \emph{Artistic
interests}, \emph{Wish to analyze}, \emph{Willingness to learn},
\emph{Sensitivity} and \emph{Intellect.}

The items to each facet are listed in the appendix (A).

\subsubsection{Results of CFA and ESEM}\label{results-of-cfa-and-esem}

All measurement models for the facets fitted well, results are
summarized in \emph{Table 3}. In this table both models with five items
only and models with all items are presented with their respective model
fit. The 5-item facets normally outperform the models including all
items regarding model fit.

\vspace{5mm}

\textless{} Table 3 here caption=\enquote{Model fit for each
facet})\textgreater{}

\vspace{5mm}

The ESEM of the final model with all five domains yielded an acceptable
fit (Marsh et al., 2010): CFI = .87, RMSEA = .072, SRMR = .036. As it
can be seen in \emph{Table 4} nearly all facets loaded significantly on
their intended domain. Some cross loadings emerged as is typical for
ESEM procedures.

\vspace{5mm}

\textless{} Table 4 here caption=\enquote{ESEM factor
scores})\textgreater{}

\vspace{5mm}

\subsubsection{Reliability}\label{reliability-1}

Reliabilities for the 5 item facets were calculated with \(\alpha\) and
\(\omega\) estimates. Agreeableness showed a mean \(\alpha\) of 0.68,
and a mean \(\omega\) of 0.69. Conscientiousness' mean \(\alpha\) =
0.68, and mean \(\omega\) = 0.70. Openness' mean \(\alpha\) = 0.76, and
mean \(\omega\) = 0.77. Neuroticism mean \(\alpha\) = 0.68, and mean
\(\omega\) = 0.69. Extraversion's mean \(\alpha\) = 0.72, and mean
\(\omega\) = 0.74.

\subsubsection{Criterion validity
evidence}\label{criterion-validity-evidence-1}

\begin{verbatim}
## Start:  AIC=2466.19
## lifesat ~ sumsA1 + sumsA2 + sumsA3 + sumsA4 + sumsA5 + sumsA6 + 
##     sumsA7 + sumsA8 + sumsC1 + sumsC2 + sumsC3 + sumsC4 + sumsC5 + 
##     sumsC6 + sumsC7 + sumsC8 + sumsC9 + sumsE1 + sumsE2 + sumsE3 + 
##     sumsE4 + sumsE5 + sumsE6 + sumsE7 + sumsE8 + sumsE9 + sumsN1 + 
##     sumsN2 + sumsN3 + sumsN4 + sumsN5 + sumsN6 + sumsN7 + sumsO1 + 
##     sumsO2 + sumsO3 + sumsO4 + sumsO5 + sumsO6 + sumsO7 + sumsO8 + 
##     sumsO9
## 
##          Df Sum of Sq   RSS    AIC
## - sumsA2  1      0.10 19510 2464.2
## - sumsN6  1      0.32 19510 2464.2
## - sumsE8  1      0.65 19511 2464.2
## - sumsN3  1      0.82 19511 2464.2
## - sumsE7  1      1.31 19512 2464.2
## - sumsA6  1      1.86 19512 2464.3
## - sumsC9  1      3.67 19514 2464.3
## - sumsA3  1      4.48 19515 2464.4
## - sumsO3  1      4.81 19515 2464.4
## - sumsA7  1      9.99 19520 2464.6
## - sumsC8  1     10.68 19521 2464.6
## - sumsO5  1     10.73 19521 2464.6
## - sumsE1  1     11.24 19521 2464.6
## - sumsA5  1     11.32 19522 2464.6
## - sumsA4  1     12.34 19522 2464.7
## - sumsC2  1     12.59 19523 2464.7
## - sumsO6  1     13.66 19524 2464.7
## - sumsE6  1     14.84 19525 2464.7
## - sumsO7  1     15.52 19526 2464.8
## - sumsN7  1     22.61 19533 2465.0
## - sumsC5  1     24.51 19535 2465.1
## - sumsN1  1     27.21 19537 2465.2
## - sumsE3  1     29.71 19540 2465.3
## - sumsC3  1     30.80 19541 2465.3
## - sumsO4  1     36.90 19547 2465.6
## - sumsA8  1     41.38 19552 2465.7
## - sumsN5  1     43.34 19554 2465.8
## - sumsC1  1     53.06 19563 2466.2
## <none>                19510 2466.2
## - sumsO1  1     55.62 19566 2466.2
## - sumsE5  1     57.49 19568 2466.3
## - sumsO8  1     68.83 19579 2466.7
## - sumsO9  1     73.73 19584 2466.9
## - sumsN4  1     77.45 19588 2467.1
## - sumsA1  1     90.84 19601 2467.6
## - sumsO2  1     96.18 19606 2467.7
## - sumsE9  1     96.74 19607 2467.8
## - sumsC4  1    108.60 19619 2468.2
## - sumsC6  1    154.37 19665 2469.9
## - sumsE2  1    185.36 19696 2471.0
## - sumsC7  1    197.28 19708 2471.5
## - sumsE4  1   1195.21 20705 2507.1
## - sumsN2  1   1959.02 21469 2533.3
## 
## Step:  AIC=2464.2
## lifesat ~ sumsA1 + sumsA3 + sumsA4 + sumsA5 + sumsA6 + sumsA7 + 
##     sumsA8 + sumsC1 + sumsC2 + sumsC3 + sumsC4 + sumsC5 + sumsC6 + 
##     sumsC7 + sumsC8 + sumsC9 + sumsE1 + sumsE2 + sumsE3 + sumsE4 + 
##     sumsE5 + sumsE6 + sumsE7 + sumsE8 + sumsE9 + sumsN1 + sumsN2 + 
##     sumsN3 + sumsN4 + sumsN5 + sumsN6 + sumsN7 + sumsO1 + sumsO2 + 
##     sumsO3 + sumsO4 + sumsO5 + sumsO6 + sumsO7 + sumsO8 + sumsO9
## 
##          Df Sum of Sq   RSS    AIC
## - sumsN6  1      0.40 19511 2462.2
## - sumsE8  1      0.66 19511 2462.2
## - sumsN3  1      0.83 19511 2462.2
## - sumsE7  1      1.37 19512 2462.2
## - sumsA6  1      2.01 19512 2462.3
## - sumsC9  1      3.60 19514 2462.3
## - sumsA3  1      4.73 19515 2462.4
## - sumsO3  1      4.78 19515 2462.4
## - sumsC8  1     10.75 19521 2462.6
## - sumsO5  1     10.80 19521 2462.6
## - sumsE1  1     11.17 19522 2462.6
## - sumsA7  1     11.20 19522 2462.6
## - sumsA5  1     11.23 19522 2462.6
## - sumsA4  1     12.59 19523 2462.7
## - sumsC2  1     12.87 19523 2462.7
## - sumsO6  1     13.57 19524 2462.7
## - sumsE6  1     15.13 19525 2462.8
## - sumsO7  1     15.78 19526 2462.8
## - sumsN7  1     22.52 19533 2463.0
## - sumsC5  1     24.44 19535 2463.1
## - sumsN1  1     27.74 19538 2463.2
## - sumsE3  1     29.66 19540 2463.3
## - sumsC3  1     30.81 19541 2463.3
## - sumsO4  1     37.24 19548 2463.6
## - sumsA8  1     41.53 19552 2463.7
## - sumsN5  1     44.67 19555 2463.8
## - sumsC1  1     53.25 19564 2464.2
## <none>                19510 2464.2
## - sumsO1  1     55.63 19566 2464.2
## - sumsE5  1     57.75 19568 2464.3
## - sumsO8  1     68.73 19579 2464.7
## - sumsO9  1     73.64 19584 2464.9
## - sumsN4  1     77.44 19588 2465.1
## - sumsA1  1     92.54 19603 2465.6
## - sumsE9  1     96.95 19607 2465.8
## - sumsO2  1     99.15 19609 2465.9
## - sumsC4  1    108.74 19619 2466.2
## - sumsC6  1    154.86 19665 2467.9
## - sumsE2  1    189.02 19699 2469.2
## - sumsC7  1    199.12 19709 2469.5
## - sumsE4  1   1200.40 20711 2505.3
## - sumsN2  1   1960.48 21471 2531.3
## 
## Step:  AIC=2462.21
## lifesat ~ sumsA1 + sumsA3 + sumsA4 + sumsA5 + sumsA6 + sumsA7 + 
##     sumsA8 + sumsC1 + sumsC2 + sumsC3 + sumsC4 + sumsC5 + sumsC6 + 
##     sumsC7 + sumsC8 + sumsC9 + sumsE1 + sumsE2 + sumsE3 + sumsE4 + 
##     sumsE5 + sumsE6 + sumsE7 + sumsE8 + sumsE9 + sumsN1 + sumsN2 + 
##     sumsN3 + sumsN4 + sumsN5 + sumsN7 + sumsO1 + sumsO2 + sumsO3 + 
##     sumsO4 + sumsO5 + sumsO6 + sumsO7 + sumsO8 + sumsO9
## 
##          Df Sum of Sq   RSS    AIC
## - sumsE8  1      0.69 19511 2460.2
## - sumsN3  1      1.12 19512 2460.2
## - sumsE7  1      1.30 19512 2460.3
## - sumsA6  1      1.86 19513 2460.3
## - sumsC9  1      3.86 19514 2460.3
## - sumsO3  1      4.66 19515 2460.4
## - sumsA3  1      5.14 19516 2460.4
## - sumsO5  1     10.76 19522 2460.6
## - sumsC8  1     10.79 19522 2460.6
## - sumsE1  1     11.14 19522 2460.6
## - sumsA7  1     11.18 19522 2460.6
## - sumsA4  1     12.64 19523 2460.7
## - sumsC2  1     12.70 19523 2460.7
## - sumsA5  1     13.50 19524 2460.7
## - sumsO6  1     13.66 19524 2460.7
## - sumsE6  1     14.80 19526 2460.8
## - sumsO7  1     15.76 19526 2460.8
## - sumsN7  1     23.63 19534 2461.1
## - sumsC5  1     24.06 19535 2461.1
## - sumsN1  1     28.25 19539 2461.3
## - sumsE3  1     29.53 19540 2461.3
## - sumsC3  1     31.60 19542 2461.4
## - sumsO4  1     37.19 19548 2461.6
## - sumsA8  1     41.16 19552 2461.7
## - sumsN5  1     44.98 19556 2461.9
## - sumsC1  1     53.08 19564 2462.2
## <none>                19511 2462.2
## - sumsO1  1     55.35 19566 2462.3
## - sumsE5  1     57.39 19568 2462.3
## - sumsO8  1     68.53 19579 2462.7
## - sumsO9  1     73.80 19584 2462.9
## - sumsN4  1     77.55 19588 2463.1
## - sumsA1  1     94.88 19606 2463.7
## - sumsE9  1     96.63 19607 2463.8
## - sumsO2  1     99.72 19610 2463.9
## - sumsC4  1    108.34 19619 2464.2
## - sumsC6  1    156.31 19667 2466.0
## - sumsE2  1    190.48 19701 2467.2
## - sumsC7  1    199.09 19710 2467.5
## - sumsE4  1   1215.84 20726 2503.9
## - sumsN2  1   2000.01 21511 2530.7
## 
## Step:  AIC=2460.24
## lifesat ~ sumsA1 + sumsA3 + sumsA4 + sumsA5 + sumsA6 + sumsA7 + 
##     sumsA8 + sumsC1 + sumsC2 + sumsC3 + sumsC4 + sumsC5 + sumsC6 + 
##     sumsC7 + sumsC8 + sumsC9 + sumsE1 + sumsE2 + sumsE3 + sumsE4 + 
##     sumsE5 + sumsE6 + sumsE7 + sumsE9 + sumsN1 + sumsN2 + sumsN3 + 
##     sumsN4 + sumsN5 + sumsN7 + sumsO1 + sumsO2 + sumsO3 + sumsO4 + 
##     sumsO5 + sumsO6 + sumsO7 + sumsO8 + sumsO9
## 
##          Df Sum of Sq   RSS    AIC
## - sumsN3  1      1.15 19512 2458.3
## - sumsE7  1      1.23 19513 2458.3
## - sumsA6  1      1.45 19513 2458.3
## - sumsC9  1      4.07 19515 2458.4
## - sumsO3  1      4.47 19516 2458.4
## - sumsA3  1      5.03 19516 2458.4
## - sumsO5  1     10.36 19522 2458.6
## - sumsC8  1     11.00 19522 2458.6
## - sumsA7  1     11.67 19523 2458.7
## - sumsA4  1     13.03 19524 2458.7
## - sumsE1  1     13.20 19525 2458.7
## - sumsO6  1     13.27 19525 2458.7
## - sumsA5  1     13.27 19525 2458.7
## - sumsC2  1     13.39 19525 2458.7
## - sumsE6  1     14.95 19526 2458.8
## - sumsO7  1     15.73 19527 2458.8
## - sumsN7  1     23.38 19535 2459.1
## - sumsC5  1     23.84 19535 2459.1
## - sumsN1  1     28.19 19540 2459.3
## - sumsE3  1     28.92 19540 2459.3
## - sumsC3  1     31.15 19542 2459.4
## - sumsO4  1     36.64 19548 2459.6
## - sumsA8  1     41.73 19553 2459.8
## - sumsN5  1     44.40 19556 2459.9
## - sumsC1  1     53.45 19565 2460.2
## <none>                19511 2460.2
## - sumsO1  1     55.97 19567 2460.3
## - sumsE5  1     56.72 19568 2460.3
## - sumsO8  1     68.00 19579 2460.8
## - sumsO9  1     73.48 19585 2460.9
## - sumsN4  1     77.21 19589 2461.1
## - sumsA1  1     94.22 19606 2461.7
## - sumsE9  1     96.02 19607 2461.8
## - sumsO2  1    100.21 19612 2461.9
## - sumsC4  1    112.77 19624 2462.4
## - sumsC6  1    158.51 19670 2464.1
## - sumsE2  1    191.36 19703 2465.3
## - sumsC7  1    198.86 19710 2465.6
## - sumsE4  1   1232.66 20744 2502.5
## - sumsN2  1   2004.87 21516 2528.9
## 
## Step:  AIC=2458.28
## lifesat ~ sumsA1 + sumsA3 + sumsA4 + sumsA5 + sumsA6 + sumsA7 + 
##     sumsA8 + sumsC1 + sumsC2 + sumsC3 + sumsC4 + sumsC5 + sumsC6 + 
##     sumsC7 + sumsC8 + sumsC9 + sumsE1 + sumsE2 + sumsE3 + sumsE4 + 
##     sumsE5 + sumsE6 + sumsE7 + sumsE9 + sumsN1 + sumsN2 + sumsN4 + 
##     sumsN5 + sumsN7 + sumsO1 + sumsO2 + sumsO3 + sumsO4 + sumsO5 + 
##     sumsO6 + sumsO7 + sumsO8 + sumsO9
## 
##          Df Sum of Sq   RSS    AIC
## - sumsE7  1      1.00 19514 2456.3
## - sumsA6  1      1.35 19514 2456.3
## - sumsC9  1      4.02 19517 2456.4
## - sumsO3  1      4.47 19517 2456.4
## - sumsA3  1      5.10 19518 2456.5
## - sumsO5  1     10.22 19523 2456.7
## - sumsA7  1     11.58 19524 2456.7
## - sumsC8  1     11.59 19524 2456.7
## - sumsA4  1     11.95 19524 2456.7
## - sumsO6  1     12.80 19525 2456.8
## - sumsE1  1     13.38 19526 2456.8
## - sumsC2  1     13.54 19526 2456.8
## - sumsA5  1     14.85 19527 2456.8
## - sumsE6  1     14.92 19528 2456.8
## - sumsO7  1     16.16 19529 2456.9
## - sumsC5  1     23.01 19536 2457.1
## - sumsN7  1     24.60 19537 2457.2
## - sumsN1  1     28.89 19541 2457.3
## - sumsE3  1     29.04 19542 2457.3
## - sumsC3  1     34.33 19547 2457.6
## - sumsO4  1     36.60 19549 2457.6
## - sumsA8  1     41.70 19554 2457.8
## - sumsN5  1     43.30 19556 2457.9
## - sumsC1  1     53.54 19566 2458.3
## <none>                19512 2458.3
## - sumsO1  1     56.07 19569 2458.3
## - sumsE5  1     56.48 19569 2458.4
## - sumsO8  1     68.24 19581 2458.8
## - sumsO9  1     73.10 19586 2459.0
## - sumsN4  1     76.09 19589 2459.1
## - sumsE9  1     95.53 19608 2459.8
## - sumsA1  1     96.20 19609 2459.8
## - sumsO2  1    100.35 19613 2460.0
## - sumsC4  1    112.03 19625 2460.4
## - sumsC6  1    157.47 19670 2462.1
## - sumsE2  1    192.79 19705 2463.4
## - sumsC7  1    197.84 19710 2463.6
## - sumsE4  1   1238.40 20751 2500.7
## - sumsN2  1   2280.65 21793 2536.1
## 
## Step:  AIC=2456.32
## lifesat ~ sumsA1 + sumsA3 + sumsA4 + sumsA5 + sumsA6 + sumsA7 + 
##     sumsA8 + sumsC1 + sumsC2 + sumsC3 + sumsC4 + sumsC5 + sumsC6 + 
##     sumsC7 + sumsC8 + sumsC9 + sumsE1 + sumsE2 + sumsE3 + sumsE4 + 
##     sumsE5 + sumsE6 + sumsE9 + sumsN1 + sumsN2 + sumsN4 + sumsN5 + 
##     sumsN7 + sumsO1 + sumsO2 + sumsO3 + sumsO4 + sumsO5 + sumsO6 + 
##     sumsO7 + sumsO8 + sumsO9
## 
##          Df Sum of Sq   RSS    AIC
## - sumsA6  1      1.23 19515 2454.4
## - sumsC9  1      4.08 19518 2454.5
## - sumsO3  1      4.97 19518 2454.5
## - sumsA3  1      5.12 19519 2454.5
## - sumsO5  1     10.24 19524 2454.7
## - sumsC8  1     10.96 19524 2454.7
## - sumsA7  1     11.03 19525 2454.7
## - sumsA4  1     12.32 19526 2454.8
## - sumsO6  1     12.56 19526 2454.8
## - sumsE1  1     12.72 19526 2454.8
## - sumsC2  1     13.59 19527 2454.8
## - sumsE6  1     14.42 19528 2454.8
## - sumsA5  1     14.55 19528 2454.9
## - sumsO7  1     16.12 19530 2454.9
## - sumsC5  1     23.80 19537 2455.2
## - sumsN7  1     24.40 19538 2455.2
## - sumsN1  1     28.66 19542 2455.4
## - sumsE3  1     29.88 19543 2455.4
## - sumsO4  1     36.02 19550 2455.7
## - sumsC3  1     36.63 19550 2455.7
## - sumsA8  1     41.89 19555 2455.9
## - sumsN5  1     42.74 19556 2455.9
## - sumsC1  1     54.13 19568 2456.3
## <none>                19514 2456.3
## - sumsE5  1     56.95 19570 2456.4
## - sumsO1  1     58.86 19572 2456.5
## - sumsO8  1     69.11 19583 2456.9
## - sumsO9  1     73.70 19587 2457.0
## - sumsN4  1     76.00 19590 2457.1
## - sumsE9  1     96.09 19610 2457.9
## - sumsA1  1     98.60 19612 2458.0
## - sumsO2  1    100.97 19614 2458.0
## - sumsC4  1    111.15 19625 2458.4
## - sumsC6  1    156.77 19670 2460.1
## - sumsE2  1    191.86 19705 2461.4
## - sumsC7  1    199.00 19712 2461.6
## - sumsE4  1   1281.15 20795 2500.2
## - sumsN2  1   2284.74 21798 2534.3
## 
## Step:  AIC=2454.36
## lifesat ~ sumsA1 + sumsA3 + sumsA4 + sumsA5 + sumsA7 + sumsA8 + 
##     sumsC1 + sumsC2 + sumsC3 + sumsC4 + sumsC5 + sumsC6 + sumsC7 + 
##     sumsC8 + sumsC9 + sumsE1 + sumsE2 + sumsE3 + sumsE4 + sumsE5 + 
##     sumsE6 + sumsE9 + sumsN1 + sumsN2 + sumsN4 + sumsN5 + sumsN7 + 
##     sumsO1 + sumsO2 + sumsO3 + sumsO4 + sumsO5 + sumsO6 + sumsO7 + 
##     sumsO8 + sumsO9
## 
##          Df Sum of Sq   RSS    AIC
## - sumsC9  1      4.40 19519 2452.5
## - sumsO3  1      5.36 19520 2452.6
## - sumsA3  1      5.47 19520 2452.6
## - sumsO5  1      9.98 19525 2452.7
## - sumsC8  1     10.56 19525 2452.8
## - sumsA7  1     10.70 19526 2452.8
## - sumsA4  1     11.20 19526 2452.8
## - sumsE1  1     12.38 19527 2452.8
## - sumsO6  1     13.18 19528 2452.8
## - sumsC2  1     13.28 19528 2452.8
## - sumsE6  1     13.98 19529 2452.9
## - sumsA5  1     14.90 19530 2452.9
## - sumsO7  1     15.69 19530 2452.9
## - sumsC5  1     22.77 19538 2453.2
## - sumsN7  1     24.08 19539 2453.2
## - sumsN1  1     27.53 19542 2453.4
## - sumsE3  1     29.93 19545 2453.5
## - sumsO4  1     35.72 19550 2453.7
## - sumsC3  1     37.98 19553 2453.8
## - sumsA8  1     40.74 19556 2453.9
## - sumsN5  1     42.09 19557 2453.9
## <none>                19515 2454.4
## - sumsC1  1     54.37 19569 2454.4
## - sumsE5  1     56.73 19572 2454.5
## - sumsO1  1     58.46 19573 2454.5
## - sumsO8  1     67.87 19583 2454.9
## - sumsO9  1     73.36 19588 2455.1
## - sumsN4  1     75.82 19591 2455.2
## - sumsE9  1     96.40 19611 2455.9
## - sumsA1  1    102.51 19617 2456.2
## - sumsO2  1    103.52 19618 2456.2
## - sumsC4  1    112.84 19628 2456.5
## - sumsC6  1    158.69 19673 2458.2
## - sumsE2  1    196.02 19711 2459.6
## - sumsC7  1    200.36 19715 2459.7
## - sumsE4  1   1314.30 20829 2499.4
## - sumsN2  1   2287.81 21803 2532.4
## 
## Step:  AIC=2452.53
## lifesat ~ sumsA1 + sumsA3 + sumsA4 + sumsA5 + sumsA7 + sumsA8 + 
##     sumsC1 + sumsC2 + sumsC3 + sumsC4 + sumsC5 + sumsC6 + sumsC7 + 
##     sumsC8 + sumsE1 + sumsE2 + sumsE3 + sumsE4 + sumsE5 + sumsE6 + 
##     sumsE9 + sumsN1 + sumsN2 + sumsN4 + sumsN5 + sumsN7 + sumsO1 + 
##     sumsO2 + sumsO3 + sumsO4 + sumsO5 + sumsO6 + sumsO7 + sumsO8 + 
##     sumsO9
## 
##          Df Sum of Sq   RSS    AIC
## - sumsA3  1      5.68 19525 2450.7
## - sumsO3  1      6.33 19526 2450.8
## - sumsO5  1     10.21 19529 2450.9
## - sumsA4  1     10.62 19530 2450.9
## - sumsC8  1     11.28 19530 2450.9
## - sumsA7  1     11.72 19531 2451.0
## - sumsE1  1     11.87 19531 2451.0
## - sumsO7  1     14.34 19534 2451.1
## - sumsC2  1     14.54 19534 2451.1
## - sumsE6  1     14.80 19534 2451.1
## - sumsA5  1     15.12 19534 2451.1
## - sumsO6  1     15.15 19534 2451.1
## - sumsC5  1     21.42 19541 2451.3
## - sumsN7  1     26.04 19545 2451.5
## - sumsE3  1     29.25 19548 2451.6
## - sumsN1  1     29.84 19549 2451.6
## - sumsO4  1     32.45 19552 2451.7
## - sumsC3  1     37.38 19556 2451.9
## - sumsA8  1     38.43 19558 2451.9
## - sumsN5  1     45.51 19565 2452.2
## <none>                19519 2452.5
## - sumsC1  1     56.12 19575 2452.6
## - sumsO1  1     58.72 19578 2452.7
## - sumsE5  1     59.38 19578 2452.7
## - sumsO9  1     69.01 19588 2453.1
## - sumsO8  1     69.56 19589 2453.1
## - sumsN4  1     78.48 19598 2453.4
## - sumsA1  1    101.38 19620 2454.3
## - sumsO2  1    101.66 19621 2454.3
## - sumsE9  1    103.45 19623 2454.3
## - sumsC4  1    108.87 19628 2454.5
## - sumsC6  1    158.38 19678 2456.4
## - sumsE2  1    194.14 19713 2457.7
## - sumsC7  1    207.79 19727 2458.2
## - sumsE4  1   1309.91 20829 2497.4
## - sumsN2  1   2284.90 21804 2530.4
## 
## Step:  AIC=2450.74
## lifesat ~ sumsA1 + sumsA4 + sumsA5 + sumsA7 + sumsA8 + sumsC1 + 
##     sumsC2 + sumsC3 + sumsC4 + sumsC5 + sumsC6 + sumsC7 + sumsC8 + 
##     sumsE1 + sumsE2 + sumsE3 + sumsE4 + sumsE5 + sumsE6 + sumsE9 + 
##     sumsN1 + sumsN2 + sumsN4 + sumsN5 + sumsN7 + sumsO1 + sumsO2 + 
##     sumsO3 + sumsO4 + sumsO5 + sumsO6 + sumsO7 + sumsO8 + sumsO9
## 
##          Df Sum of Sq   RSS    AIC
## - sumsO3  1      6.46 19531 2449.0
## - sumsE1  1     10.10 19535 2449.1
## - sumsO5  1     11.04 19536 2449.1
## - sumsC8  1     12.00 19537 2449.2
## - sumsA7  1     13.53 19538 2449.2
## - sumsA4  1     13.53 19538 2449.2
## - sumsO7  1     14.32 19539 2449.3
## - sumsA5  1     14.55 19539 2449.3
## - sumsC2  1     14.60 19539 2449.3
## - sumsO6  1     15.13 19540 2449.3
## - sumsE6  1     18.61 19544 2449.4
## - sumsC5  1     21.29 19546 2449.5
## - sumsN7  1     23.46 19548 2449.6
## - sumsE3  1     27.38 19552 2449.8
## - sumsN1  1     29.75 19555 2449.8
## - sumsO4  1     31.07 19556 2449.9
## - sumsA8  1     34.73 19560 2450.0
## - sumsC3  1     37.76 19563 2450.1
## - sumsN5  1     44.93 19570 2450.4
## <none>                19525 2450.7
## - sumsE5  1     54.69 19580 2450.8
## - sumsO1  1     57.99 19583 2450.9
## - sumsO8  1     67.03 19592 2451.2
## - sumsO9  1     70.51 19595 2451.3
## - sumsN4  1     74.87 19600 2451.5
## - sumsC1  1     82.23 19607 2451.8
## - sumsO2  1    102.59 19627 2452.5
## - sumsA1  1    103.30 19628 2452.6
## - sumsE9  1    104.84 19630 2452.6
## - sumsC4  1    111.01 19636 2452.8
## - sumsC6  1    159.60 19684 2454.6
## - sumsE2  1    196.01 19721 2455.9
## - sumsC7  1    211.30 19736 2456.5
## - sumsE4  1   1308.04 20833 2495.6
## - sumsN2  1   2282.09 21807 2528.6
## 
## Step:  AIC=2448.97
## lifesat ~ sumsA1 + sumsA4 + sumsA5 + sumsA7 + sumsA8 + sumsC1 + 
##     sumsC2 + sumsC3 + sumsC4 + sumsC5 + sumsC6 + sumsC7 + sumsC8 + 
##     sumsE1 + sumsE2 + sumsE3 + sumsE4 + sumsE5 + sumsE6 + sumsE9 + 
##     sumsN1 + sumsN2 + sumsN4 + sumsN5 + sumsN7 + sumsO1 + sumsO2 + 
##     sumsO4 + sumsO5 + sumsO6 + sumsO7 + sumsO8 + sumsO9
## 
##          Df Sum of Sq   RSS    AIC
## - sumsE1  1      9.54 19541 2447.3
## - sumsO5  1     11.67 19543 2447.4
## - sumsO6  1     12.56 19544 2447.4
## - sumsC2  1     13.22 19544 2447.5
## - sumsA4  1     13.40 19545 2447.5
## - sumsC8  1     14.16 19546 2447.5
## - sumsO7  1     14.21 19546 2447.5
## - sumsA7  1     14.34 19546 2447.5
## - sumsA5  1     15.19 19546 2447.5
## - sumsE6  1     18.79 19550 2447.7
## - sumsC5  1     21.91 19553 2447.8
## - sumsN7  1     22.82 19554 2447.8
## - sumsE3  1     28.37 19560 2448.0
## - sumsO4  1     29.53 19561 2448.1
## - sumsN1  1     29.57 19561 2448.1
## - sumsA8  1     33.98 19565 2448.2
## - sumsC3  1     37.09 19568 2448.3
## - sumsN5  1     44.74 19576 2448.6
## <none>                19531 2449.0
## - sumsE5  1     60.92 19592 2449.2
## - sumsO9  1     65.12 19596 2449.4
## - sumsO8  1     69.20 19600 2449.5
## - sumsO1  1     70.84 19602 2449.6
## - sumsN4  1     83.16 19614 2450.0
## - sumsC1  1     84.90 19616 2450.1
## - sumsO2  1    101.54 19633 2450.7
## - sumsE9  1    108.61 19640 2451.0
## - sumsC4  1    112.76 19644 2451.1
## - sumsA1  1    133.80 19665 2451.9
## - sumsC6  1    153.41 19685 2452.6
## - sumsE2  1    200.74 19732 2454.4
## - sumsC7  1    211.47 19743 2454.8
## - sumsE4  1   1312.31 20844 2493.9
## - sumsN2  1   2281.22 21812 2526.7
## 
## Step:  AIC=2447.33
## lifesat ~ sumsA1 + sumsA4 + sumsA5 + sumsA7 + sumsA8 + sumsC1 + 
##     sumsC2 + sumsC3 + sumsC4 + sumsC5 + sumsC6 + sumsC7 + sumsC8 + 
##     sumsE2 + sumsE3 + sumsE4 + sumsE5 + sumsE6 + sumsE9 + sumsN1 + 
##     sumsN2 + sumsN4 + sumsN5 + sumsN7 + sumsO1 + sumsO2 + sumsO4 + 
##     sumsO5 + sumsO6 + sumsO7 + sumsO8 + sumsO9
## 
##          Df Sum of Sq   RSS    AIC
## - sumsO5  1     10.59 19551 2445.7
## - sumsO6  1     10.79 19552 2445.7
## - sumsC2  1     12.87 19554 2445.8
## - sumsO7  1     13.43 19554 2445.8
## - sumsC8  1     13.76 19555 2445.8
## - sumsA4  1     14.05 19555 2445.8
## - sumsA5  1     14.53 19555 2445.9
## - sumsA7  1     16.89 19558 2445.9
## - sumsC5  1     21.98 19563 2446.1
## - sumsE3  1     22.38 19563 2446.2
## - sumsE6  1     24.65 19566 2446.2
## - sumsN7  1     25.28 19566 2446.3
## - sumsO4  1     29.81 19571 2446.4
## - sumsA8  1     33.10 19574 2446.6
## - sumsN1  1     33.82 19575 2446.6
## - sumsC3  1     38.07 19579 2446.7
## - sumsN5  1     42.74 19584 2446.9
## <none>                19541 2447.3
## - sumsE5  1     54.28 19595 2447.3
## - sumsO9  1     65.43 19606 2447.7
## - sumsO1  1     69.63 19610 2447.9
## - sumsN4  1     77.76 19619 2448.2
## - sumsC1  1     81.63 19622 2448.3
## - sumsO8  1     83.29 19624 2448.4
## - sumsO2  1    103.56 19644 2449.1
## - sumsE9  1    105.66 19646 2449.2
## - sumsC4  1    113.95 19655 2449.5
## - sumsA1  1    134.30 19675 2450.3
## - sumsC6  1    150.47 19691 2450.9
## - sumsE2  1    203.41 19744 2452.8
## - sumsC7  1    209.00 19750 2453.0
## - sumsE4  1   1311.16 20852 2492.2
## - sumsN2  1   2316.17 21857 2526.2
## 
## Step:  AIC=2445.72
## lifesat ~ sumsA1 + sumsA4 + sumsA5 + sumsA7 + sumsA8 + sumsC1 + 
##     sumsC2 + sumsC3 + sumsC4 + sumsC5 + sumsC6 + sumsC7 + sumsC8 + 
##     sumsE2 + sumsE3 + sumsE4 + sumsE5 + sumsE6 + sumsE9 + sumsN1 + 
##     sumsN2 + sumsN4 + sumsN5 + sumsN7 + sumsO1 + sumsO2 + sumsO4 + 
##     sumsO6 + sumsO7 + sumsO8 + sumsO9
## 
##          Df Sum of Sq   RSS    AIC
## - sumsO6  1      9.34 19561 2444.1
## - sumsC2  1     11.78 19563 2444.2
## - sumsC8  1     15.33 19567 2444.3
## - sumsO7  1     15.53 19567 2444.3
## - sumsA4  1     15.86 19567 2444.3
## - sumsA5  1     16.04 19568 2444.3
## - sumsA7  1     16.46 19568 2444.3
## - sumsO4  1     21.22 19573 2444.5
## - sumsE3  1     22.15 19574 2444.5
## - sumsC5  1     23.00 19574 2444.6
## - sumsN7  1     23.41 19575 2444.6
## - sumsE6  1     26.46 19578 2444.7
## - sumsA8  1     30.40 19582 2444.8
## - sumsC3  1     34.15 19586 2445.0
## - sumsN1  1     35.23 19587 2445.0
## - sumsN5  1     42.81 19594 2445.3
## - sumsE5  1     52.40 19604 2445.7
## <none>                19551 2445.7
## - sumsO9  1     67.88 19619 2446.2
## - sumsN4  1     77.09 19628 2446.6
## - sumsO1  1     77.52 19629 2446.6
## - sumsO8  1     79.74 19631 2446.7
## - sumsC1  1     83.99 19635 2446.8
## - sumsE9  1    105.80 19657 2447.6
## - sumsC4  1    118.53 19670 2448.1
## - sumsO2  1    124.99 19676 2448.3
## - sumsA1  1    136.55 19688 2448.7
## - sumsC6  1    152.87 19704 2449.3
## - sumsE2  1    201.09 19752 2451.1
## - sumsC7  1    208.97 19760 2451.4
## - sumsE4  1   1301.70 20853 2490.3
## - sumsN2  1   2377.53 21929 2526.6
## 
## Step:  AIC=2444.06
## lifesat ~ sumsA1 + sumsA4 + sumsA5 + sumsA7 + sumsA8 + sumsC1 + 
##     sumsC2 + sumsC3 + sumsC4 + sumsC5 + sumsC6 + sumsC7 + sumsC8 + 
##     sumsE2 + sumsE3 + sumsE4 + sumsE5 + sumsE6 + sumsE9 + sumsN1 + 
##     sumsN2 + sumsN4 + sumsN5 + sumsN7 + sumsO1 + sumsO2 + sumsO4 + 
##     sumsO7 + sumsO8 + sumsO9
## 
##          Df Sum of Sq   RSS    AIC
## - sumsO7  1     10.38 19571 2442.4
## - sumsC2  1     12.06 19573 2442.5
## - sumsA4  1     14.40 19575 2442.6
## - sumsC8  1     15.23 19576 2442.6
## - sumsA5  1     15.30 19576 2442.6
## - sumsA7  1     18.22 19579 2442.7
## - sumsN7  1     21.13 19582 2442.8
## - sumsC5  1     22.83 19584 2442.9
## - sumsO4  1     23.27 19584 2442.9
## - sumsE3  1     24.99 19586 2443.0
## - sumsE6  1     27.69 19588 2443.1
## - sumsA8  1     29.32 19590 2443.1
## - sumsC3  1     30.98 19592 2443.2
## - sumsN1  1     39.72 19600 2443.5
## - sumsN5  1     44.08 19605 2443.7
## - sumsE5  1     48.33 19609 2443.8
## <none>                19561 2444.1
## - sumsO1  1     72.45 19633 2444.7
## - sumsN4  1     75.88 19637 2444.9
## - sumsO9  1     75.88 19637 2444.9
## - sumsO8  1     83.88 19645 2445.2
## - sumsC1  1     89.03 19650 2445.3
## - sumsE9  1    109.77 19670 2446.1
## - sumsC4  1    118.19 19679 2446.4
## - sumsO2  1    120.51 19681 2446.5
## - sumsA1  1    132.04 19693 2446.9
## - sumsC6  1    183.92 19745 2448.8
## - sumsE2  1    197.96 19759 2449.3
## - sumsC7  1    211.57 19772 2449.8
## - sumsE4  1   1292.36 20853 2488.3
## - sumsN2  1   2389.32 21950 2525.3
## 
## Step:  AIC=2442.45
## lifesat ~ sumsA1 + sumsA4 + sumsA5 + sumsA7 + sumsA8 + sumsC1 + 
##     sumsC2 + sumsC3 + sumsC4 + sumsC5 + sumsC6 + sumsC7 + sumsC8 + 
##     sumsE2 + sumsE3 + sumsE4 + sumsE5 + sumsE6 + sumsE9 + sumsN1 + 
##     sumsN2 + sumsN4 + sumsN5 + sumsN7 + sumsO1 + sumsO2 + sumsO4 + 
##     sumsO8 + sumsO9
## 
##          Df Sum of Sq   RSS    AIC
## - sumsC2  1     11.85 19583 2440.9
## - sumsA5  1     13.59 19585 2440.9
## - sumsC8  1     17.22 19588 2441.1
## - sumsA4  1     17.30 19588 2441.1
## - sumsO4  1     17.43 19589 2441.1
## - sumsA7  1     19.56 19591 2441.2
## - sumsC5  1     19.77 19591 2441.2
## - sumsE3  1     22.64 19594 2441.3
## - sumsN7  1     25.24 19596 2441.4
## - sumsE6  1     28.59 19600 2441.5
## - sumsA8  1     29.13 19600 2441.5
## - sumsC3  1     30.60 19602 2441.6
## - sumsN1  1     38.63 19610 2441.9
## - sumsN5  1     46.63 19618 2442.2
## - sumsE5  1     48.43 19620 2442.2
## <none>                19571 2442.4
## - sumsO9  1     73.75 19645 2443.2
## - sumsN4  1     74.66 19646 2443.2
## - sumsO8  1     79.17 19650 2443.4
## - sumsO1  1     88.14 19659 2443.7
## - sumsC1  1     88.94 19660 2443.7
## - sumsE9  1    104.85 19676 2444.3
## - sumsC4  1    120.51 19692 2444.9
## - sumsO2  1    151.76 19723 2446.0
## - sumsA1  1    160.83 19732 2446.4
## - sumsC6  1    178.92 19750 2447.0
## - sumsE2  1    205.67 19777 2448.0
## - sumsC7  1    209.40 19781 2448.1
## - sumsE4  1   1281.98 20853 2486.3
## - sumsN2  1   2379.29 21950 2523.3
## 
## Step:  AIC=2440.88
## lifesat ~ sumsA1 + sumsA4 + sumsA5 + sumsA7 + sumsA8 + sumsC1 + 
##     sumsC3 + sumsC4 + sumsC5 + sumsC6 + sumsC7 + sumsC8 + sumsE2 + 
##     sumsE3 + sumsE4 + sumsE5 + sumsE6 + sumsE9 + sumsN1 + sumsN2 + 
##     sumsN4 + sumsN5 + sumsN7 + sumsO1 + sumsO2 + sumsO4 + sumsO8 + 
##     sumsO9
## 
##          Df Sum of Sq   RSS    AIC
## - sumsA5  1     11.37 19594 2439.3
## - sumsC5  1     14.97 19598 2439.4
## - sumsC8  1     16.95 19600 2439.5
## - sumsO4  1     18.00 19601 2439.6
## - sumsA7  1     18.63 19602 2439.6
## - sumsA4  1     19.99 19603 2439.6
## - sumsN7  1     22.91 19606 2439.7
## - sumsE3  1     26.97 19610 2439.9
## - sumsA8  1     28.42 19611 2439.9
## - sumsC3  1     29.02 19612 2439.9
## - sumsE6  1     31.97 19615 2440.1
## - sumsN1  1     37.13 19620 2440.2
## - sumsE5  1     45.60 19629 2440.6
## <none>                19583 2440.9
## - sumsN5  1     59.20 19642 2441.1
## - sumsO9  1     70.93 19654 2441.5
## - sumsN4  1     80.29 19663 2441.8
## - sumsO8  1     80.30 19663 2441.8
## - sumsC1  1     84.90 19668 2442.0
## - sumsO1  1     89.00 19672 2442.2
## - sumsE9  1    108.54 19692 2442.9
## - sumsC4  1    119.82 19703 2443.3
## - sumsO2  1    148.72 19732 2444.3
## - sumsA1  1    158.26 19741 2444.7
## - sumsC6  1    179.02 19762 2445.4
## - sumsE2  1    199.61 19783 2446.2
## - sumsC7  1    214.91 19798 2446.8
## - sumsE4  1   1282.64 20866 2484.7
## - sumsN2  1   2375.34 21958 2521.5
## 
## Step:  AIC=2439.3
## lifesat ~ sumsA1 + sumsA4 + sumsA7 + sumsA8 + sumsC1 + sumsC3 + 
##     sumsC4 + sumsC5 + sumsC6 + sumsC7 + sumsC8 + sumsE2 + sumsE3 + 
##     sumsE4 + sumsE5 + sumsE6 + sumsE9 + sumsN1 + sumsN2 + sumsN4 + 
##     sumsN5 + sumsN7 + sumsO1 + sumsO2 + sumsO4 + sumsO8 + sumsO9
## 
##          Df Sum of Sq   RSS    AIC
## - sumsC5  1     14.26 19609 2437.8
## - sumsC8  1     16.24 19611 2437.9
## - sumsO4  1     16.96 19611 2437.9
## - sumsA7  1     20.51 19615 2438.1
## - sumsA4  1     21.12 19616 2438.1
## - sumsA8  1     25.14 19620 2438.2
## - sumsE6  1     25.83 19620 2438.2
## - sumsN7  1     26.89 19621 2438.3
## - sumsE3  1     28.56 19623 2438.3
## - sumsC3  1     35.89 19630 2438.6
## - sumsE5  1     44.59 19639 2438.9
## - sumsN1  1     53.21 19648 2439.3
## <none>                19594 2439.3
## - sumsN5  1     55.42 19650 2439.3
## - sumsO8  1     70.88 19665 2439.9
## - sumsN4  1     74.91 19669 2440.1
## - sumsO9  1     77.41 19672 2440.2
## - sumsC1  1     81.54 19676 2440.3
## - sumsO1  1     88.12 19682 2440.5
## - sumsE9  1    100.71 19695 2441.0
## - sumsC4  1    117.82 19712 2441.6
## - sumsO2  1    146.18 19741 2442.7
## - sumsA1  1    157.58 19752 2443.1
## - sumsC6  1    177.67 19772 2443.8
## - sumsE2  1    204.82 19799 2444.8
## - sumsC7  1    221.51 19816 2445.4
## - sumsE4  1   1277.41 20872 2482.9
## - sumsN2  1   2441.28 22036 2522.1
## 
## Step:  AIC=2437.83
## lifesat ~ sumsA1 + sumsA4 + sumsA7 + sumsA8 + sumsC1 + sumsC3 + 
##     sumsC4 + sumsC6 + sumsC7 + sumsC8 + sumsE2 + sumsE3 + sumsE4 + 
##     sumsE5 + sumsE6 + sumsE9 + sumsN1 + sumsN2 + sumsN4 + sumsN5 + 
##     sumsN7 + sumsO1 + sumsO2 + sumsO4 + sumsO8 + sumsO9
## 
##          Df Sum of Sq   RSS    AIC
## - sumsC8  1     11.72 19620 2436.3
## - sumsO4  1     16.44 19625 2436.4
## - sumsA4  1     19.22 19628 2436.5
## - sumsA7  1     20.05 19629 2436.6
## - sumsE6  1     24.29 19633 2436.7
## - sumsA8  1     26.61 19635 2436.8
## - sumsE3  1     27.25 19636 2436.8
## - sumsN7  1     29.46 19638 2436.9
## - sumsC3  1     35.89 19644 2437.2
## - sumsE5  1     45.46 19654 2437.5
## - sumsN5  1     46.28 19655 2437.5
## - sumsN1  1     51.00 19660 2437.7
## <none>                19609 2437.8
## - sumsO8  1     72.29 19681 2438.5
## - sumsN4  1     75.48 19684 2438.6
## - sumsO1  1     84.84 19694 2438.9
## - sumsO9  1     86.60 19695 2439.0
## - sumsC1  1     88.13 19697 2439.1
## - sumsE9  1     95.14 19704 2439.3
## - sumsO2  1    133.36 19742 2440.7
## - sumsA1  1    158.80 19767 2441.7
## - sumsC4  1    166.84 19776 2441.9
## - sumsC6  1    188.34 19797 2442.7
## - sumsE2  1    196.20 19805 2443.0
## - sumsC7  1    212.74 19821 2443.6
## - sumsE4  1   1321.33 20930 2482.9
## - sumsN2  1   2431.66 22040 2520.2
## 
## Step:  AIC=2436.26
## lifesat ~ sumsA1 + sumsA4 + sumsA7 + sumsA8 + sumsC1 + sumsC3 + 
##     sumsC4 + sumsC6 + sumsC7 + sumsE2 + sumsE3 + sumsE4 + sumsE5 + 
##     sumsE6 + sumsE9 + sumsN1 + sumsN2 + sumsN4 + sumsN5 + sumsN7 + 
##     sumsO1 + sumsO2 + sumsO4 + sumsO8 + sumsO9
## 
##          Df Sum of Sq   RSS    AIC
## - sumsO4  1     17.74 19638 2434.9
## - sumsA4  1     18.00 19638 2434.9
## - sumsA7  1     22.69 19643 2435.1
## - sumsA8  1     23.61 19644 2435.1
## - sumsE6  1     25.96 19646 2435.2
## - sumsE3  1     27.98 19648 2435.3
## - sumsN7  1     28.95 19649 2435.3
## - sumsC3  1     40.24 19661 2435.7
## - sumsE5  1     47.31 19668 2436.0
## - sumsN1  1     50.55 19671 2436.1
## - sumsN5  1     52.19 19672 2436.2
## <none>                19620 2436.3
## - sumsO8  1     71.16 19692 2436.9
## - sumsN4  1     76.95 19697 2437.1
## - sumsC1  1     83.47 19704 2437.3
## - sumsO9  1     83.80 19704 2437.3
## - sumsO1  1     85.02 19705 2437.4
## - sumsE9  1     99.74 19720 2437.9
## - sumsO2  1    138.39 19759 2439.3
## - sumsC4  1    155.71 19776 2440.0
## - sumsA1  1    158.96 19779 2440.1
## - sumsC6  1    183.03 19803 2441.0
## - sumsC7  1    204.70 19825 2441.8
## - sumsE2  1    205.60 19826 2441.8
## - sumsE4  1   1317.39 20938 2481.2
## - sumsN2  1   2465.10 22086 2519.7
## 
## Step:  AIC=2434.91
## lifesat ~ sumsA1 + sumsA4 + sumsA7 + sumsA8 + sumsC1 + sumsC3 + 
##     sumsC4 + sumsC6 + sumsC7 + sumsE2 + sumsE3 + sumsE4 + sumsE5 + 
##     sumsE6 + sumsE9 + sumsN1 + sumsN2 + sumsN4 + sumsN5 + sumsN7 + 
##     sumsO1 + sumsO2 + sumsO8 + sumsO9
## 
##          Df Sum of Sq   RSS    AIC
## - sumsA4  1     17.06 19655 2433.5
## - sumsA7  1     22.21 19660 2433.7
## - sumsA8  1     25.92 19664 2433.9
## - sumsN7  1     27.02 19665 2433.9
## - sumsE3  1     30.41 19668 2434.0
## - sumsE6  1     30.82 19669 2434.0
## - sumsC3  1     41.42 19680 2434.4
## - sumsE5  1     49.09 19687 2434.7
## - sumsN1  1     51.72 19690 2434.8
## - sumsN5  1     54.11 19692 2434.9
## <none>                19638 2434.9
## - sumsO8  1     70.20 19708 2435.5
## - sumsN4  1     77.55 19716 2435.8
## - sumsO1  1     78.60 19717 2435.8
## - sumsC1  1     82.84 19721 2435.9
## - sumsE9  1     97.15 19735 2436.5
## - sumsO9  1    105.30 19743 2436.8
## - sumsO2  1    127.38 19766 2437.6
## - sumsC4  1    148.55 19787 2438.3
## - sumsA1  1    152.26 19790 2438.5
## - sumsC6  1    185.75 19824 2439.7
## - sumsE2  1    192.69 19831 2440.0
## - sumsC7  1    217.79 19856 2440.9
## - sumsE4  1   1309.79 20948 2479.5
## - sumsN2  1   2455.47 22094 2518.0
## 
## Step:  AIC=2433.54
## lifesat ~ sumsA1 + sumsA7 + sumsA8 + sumsC1 + sumsC3 + sumsC4 + 
##     sumsC6 + sumsC7 + sumsE2 + sumsE3 + sumsE4 + sumsE5 + sumsE6 + 
##     sumsE9 + sumsN1 + sumsN2 + sumsN4 + sumsN5 + sumsN7 + sumsO1 + 
##     sumsO2 + sumsO8 + sumsO9
## 
##          Df Sum of Sq   RSS    AIC
## - sumsN7  1     16.28 19671 2432.1
## - sumsA7  1     23.90 19679 2432.4
## - sumsA8  1     29.53 19685 2432.6
## - sumsE3  1     30.72 19686 2432.7
## - sumsE6  1     34.21 19689 2432.8
## - sumsC3  1     34.47 19690 2432.8
## - sumsE5  1     52.50 19708 2433.5
## <none>                19655 2433.5
## - sumsN1  1     62.29 19718 2433.8
## - sumsN5  1     63.14 19718 2433.8
## - sumsO8  1     67.28 19722 2434.0
## - sumsN4  1     80.89 19736 2434.5
## - sumsC1  1     86.21 19741 2434.7
## - sumsO1  1     92.50 19748 2434.9
## - sumsE9  1     95.09 19750 2435.0
## - sumsO9  1    100.64 19756 2435.2
## - sumsO2  1    128.18 19783 2436.2
## - sumsA1  1    140.43 19796 2436.7
## - sumsC4  1    168.22 19823 2437.7
## - sumsC6  1    194.90 19850 2438.7
## - sumsE2  1    200.07 19855 2438.8
## - sumsC7  1    214.02 19869 2439.4
## - sumsE4  1   1319.96 20975 2478.5
## - sumsN2  1   2446.34 22102 2516.2
## 
## Step:  AIC=2432.14
## lifesat ~ sumsA1 + sumsA7 + sumsA8 + sumsC1 + sumsC3 + sumsC4 + 
##     sumsC6 + sumsC7 + sumsE2 + sumsE3 + sumsE4 + sumsE5 + sumsE6 + 
##     sumsE9 + sumsN1 + sumsN2 + sumsN4 + sumsN5 + sumsO1 + sumsO2 + 
##     sumsO8 + sumsO9
## 
##          Df Sum of Sq   RSS    AIC
## - sumsA7  1     21.63 19693 2430.9
## - sumsE3  1     27.52 19699 2431.2
## - sumsE6  1     27.55 19699 2431.2
## - sumsA8  1     29.62 19701 2431.2
## - sumsE5  1     48.21 19720 2431.9
## - sumsC3  1     50.56 19722 2432.0
## <none>                19671 2432.1
## - sumsN5  1     56.95 19728 2432.2
## - sumsN1  1     61.95 19733 2432.4
## - sumsO8  1     68.52 19740 2432.7
## - sumsC1  1     75.77 19747 2432.9
## - sumsN4  1     76.08 19748 2432.9
## - sumsO1  1     85.21 19757 2433.3
## - sumsE9  1     95.95 19767 2433.7
## - sumsO9  1    102.76 19774 2433.9
## - sumsA1  1    132.60 19804 2435.0
## - sumsO2  1    134.96 19806 2435.1
## - sumsC4  1    166.59 19838 2436.2
## - sumsC6  1    183.20 19855 2436.8
## - sumsC7  1    212.00 19884 2437.9
## - sumsE2  1    217.21 19889 2438.1
## - sumsE4  1   1309.06 20980 2476.7
## - sumsN2  1   2614.88 22286 2520.2
## 
## Step:  AIC=2430.93
## lifesat ~ sumsA1 + sumsA8 + sumsC1 + sumsC3 + sumsC4 + sumsC6 + 
##     sumsC7 + sumsE2 + sumsE3 + sumsE4 + sumsE5 + sumsE6 + sumsE9 + 
##     sumsN1 + sumsN2 + sumsN4 + sumsN5 + sumsO1 + sumsO2 + sumsO8 + 
##     sumsO9
## 
##          Df Sum of Sq   RSS    AIC
## - sumsA8  1     22.84 19716 2429.8
## - sumsE6  1     27.16 19720 2429.9
## - sumsE3  1     30.13 19723 2430.0
## - sumsE5  1     38.32 19731 2430.3
## - sumsC3  1     39.20 19732 2430.4
## <none>                19693 2430.9
## - sumsN5  1     56.57 19750 2431.0
## - sumsN1  1     59.58 19753 2431.1
## - sumsO8  1     61.94 19755 2431.2
## - sumsN4  1     77.33 19770 2431.8
## - sumsO1  1     81.94 19775 2431.9
## - sumsE9  1     93.47 19787 2432.3
## - sumsC1  1     98.56 19792 2432.5
## - sumsO9  1    106.10 19799 2432.8
## - sumsA1  1    144.70 19838 2434.2
## - sumsO2  1    146.19 19839 2434.3
## - sumsC4  1    164.17 19857 2434.9
## - sumsC6  1    173.84 19867 2435.3
## - sumsC7  1    221.47 19915 2437.0
## - sumsE2  1    232.31 19925 2437.4
## - sumsE4  1   1308.79 21002 2475.4
## - sumsN2  1   2600.62 22294 2518.5
## 
## Step:  AIC=2429.77
## lifesat ~ sumsA1 + sumsC1 + sumsC3 + sumsC4 + sumsC6 + sumsC7 + 
##     sumsE2 + sumsE3 + sumsE4 + sumsE5 + sumsE6 + sumsE9 + sumsN1 + 
##     sumsN2 + sumsN4 + sumsN5 + sumsO1 + sumsO2 + sumsO8 + sumsO9
## 
##          Df Sum of Sq   RSS    AIC
## - sumsE6  1     28.39 19744 2428.8
## - sumsE3  1     31.85 19748 2428.9
## - sumsE5  1     37.97 19754 2429.2
## - sumsC3  1     39.60 19756 2429.2
## <none>                19716 2429.8
## - sumsN5  1     64.00 19780 2430.1
## - sumsO8  1     64.83 19781 2430.1
## - sumsN4  1     72.27 19788 2430.4
## - sumsN1  1     73.68 19790 2430.5
## - sumsO1  1     83.06 19799 2430.8
## - sumsE9  1     84.80 19801 2430.9
## - sumsC1  1     94.73 19811 2431.2
## - sumsO9  1    109.07 19825 2431.8
## - sumsA1  1    122.33 19838 2432.2
## - sumsO2  1    152.81 19869 2433.3
## - sumsC4  1    172.18 19888 2434.1
## - sumsC6  1    175.02 19891 2434.2
## - sumsE2  1    225.22 19941 2436.0
## - sumsC7  1    241.72 19958 2436.6
## - sumsE4  1   1354.46 21070 2475.7
## - sumsN2  1   2577.84 22294 2516.5
## 
## Step:  AIC=2428.81
## lifesat ~ sumsA1 + sumsC1 + sumsC3 + sumsC4 + sumsC6 + sumsC7 + 
##     sumsE2 + sumsE3 + sumsE4 + sumsE5 + sumsE9 + sumsN1 + sumsN2 + 
##     sumsN4 + sumsN5 + sumsO1 + sumsO2 + sumsO8 + sumsO9
## 
##          Df Sum of Sq   RSS    AIC
## - sumsE5  1     23.92 19768 2427.7
## - sumsE3  1     24.05 19768 2427.7
## - sumsC3  1     31.03 19775 2427.9
## <none>                19744 2428.8
## - sumsN1  1     67.36 19812 2429.3
## - sumsN5  1     69.67 19814 2429.3
## - sumsN4  1     73.45 19818 2429.5
## - sumsE9  1     78.71 19823 2429.7
## - sumsO1  1     88.59 19833 2430.0
## - sumsC1  1     91.33 19836 2430.1
## - sumsO9  1     99.37 19844 2430.4
## - sumsA1  1    128.88 19873 2431.5
## - sumsO8  1    138.03 19882 2431.8
## - sumsO2  1    144.54 19889 2432.1
## - sumsC6  1    163.15 19908 2432.8
## - sumsC4  1    192.22 19936 2433.8
## - sumsE2  1    215.59 19960 2434.7
## - sumsC7  1    257.67 20002 2436.2
## - sumsE4  1   1519.72 21264 2480.3
## - sumsN2  1   2550.85 22295 2514.5
## 
## Step:  AIC=2427.68
## lifesat ~ sumsA1 + sumsC1 + sumsC3 + sumsC4 + sumsC6 + sumsC7 + 
##     sumsE2 + sumsE3 + sumsE4 + sumsE9 + sumsN1 + sumsN2 + sumsN4 + 
##     sumsN5 + sumsO1 + sumsO2 + sumsO8 + sumsO9
## 
##          Df Sum of Sq   RSS    AIC
## - sumsC3  1     31.98 19800 2426.8
## - sumsE3  1     34.10 19802 2426.9
## <none>                19768 2427.7
## - sumsC1  1     67.48 19836 2428.1
## - sumsN5  1     75.39 19844 2428.4
## - sumsE9  1     77.32 19846 2428.5
## - sumsN1  1     81.22 19850 2428.6
## - sumsN4  1     81.88 19850 2428.7
## - sumsO9  1     86.12 19854 2428.8
## - sumsO1  1    121.82 19890 2430.1
## - sumsO8  1    131.92 19900 2430.5
## - sumsA1  1    136.04 19904 2430.6
## - sumsO2  1    136.98 19905 2430.7
## - sumsC6  1    165.63 19934 2431.7
## - sumsC4  1    186.39 19955 2432.5
## - sumsE2  1    208.66 19977 2433.3
## - sumsC7  1    244.58 20013 2434.6
## - sumsE4  1   1505.60 21274 2478.7
## - sumsN2  1   2559.40 22328 2513.6
## 
## Step:  AIC=2426.85
## lifesat ~ sumsA1 + sumsC1 + sumsC4 + sumsC6 + sumsC7 + sumsE2 + 
##     sumsE3 + sumsE4 + sumsE9 + sumsN1 + sumsN2 + sumsN4 + sumsN5 + 
##     sumsO1 + sumsO2 + sumsO8 + sumsO9
## 
##          Df Sum of Sq   RSS    AIC
## - sumsE3  1     37.54 19838 2426.2
## <none>                19800 2426.8
## - sumsN5  1     58.06 19858 2427.0
## - sumsC1  1     65.54 19866 2427.2
## - sumsE9  1     72.79 19873 2427.5
## - sumsN4  1     79.55 19880 2427.7
## - sumsO9  1     88.15 19888 2428.1
## - sumsN1  1    119.44 19920 2429.2
## - sumsO8  1    126.11 19926 2429.4
## - sumsO1  1    133.82 19934 2429.7
## - sumsA1  1    136.59 19937 2429.8
## - sumsO2  1    147.80 19948 2430.2
## - sumsE2  1    178.04 19978 2431.3
## - sumsC4  1    190.16 19990 2431.8
## - sumsC6  1    191.44 19992 2431.8
## - sumsC7  1    218.71 20019 2432.8
## - sumsE4  1   1474.42 21275 2476.7
## - sumsN2  1   2791.24 22592 2520.1
## 
## Step:  AIC=2426.22
## lifesat ~ sumsA1 + sumsC1 + sumsC4 + sumsC6 + sumsC7 + sumsE2 + 
##     sumsE4 + sumsE9 + sumsN1 + sumsN2 + sumsN4 + sumsN5 + sumsO1 + 
##     sumsO2 + sumsO8 + sumsO9
## 
##          Df Sum of Sq   RSS    AIC
## <none>                19838 2426.2
## - sumsC1  1     62.12 19900 2426.5
## - sumsN5  1     64.30 19902 2426.6
## - sumsN4  1     71.02 19909 2426.8
## - sumsE9  1     92.24 19930 2427.6
## - sumsO9  1     98.26 19936 2427.8
## - sumsO8  1    100.96 19939 2427.9
## - sumsN1  1    113.51 19951 2428.3
## - sumsO1  1    120.43 19958 2428.6
## - sumsA1  1    138.85 19977 2429.2
## - sumsO2  1    146.84 19985 2429.5
## - sumsE2  1    162.36 20000 2430.1
## - sumsC4  1    199.33 20037 2431.4
## - sumsC6  1    203.33 20041 2431.6
## - sumsC7  1    232.24 20070 2432.6
## - sumsE4  1   1459.93 21298 2475.5
## - sumsN2  1   2808.17 22646 2519.8
\end{verbatim}

\begin{verbatim}
## Start:  AIC=-732.71
## hsgpa_num ~ sumsO + sumsE + sumsN + sumsA + sumsC
## 
##         Df Sum of Sq    RSS     AIC
## - sumsE  1    0.0137 251.62 -734.67
## - sumsO  1    0.4334 252.04 -733.48
## <none>               251.61 -732.71
## - sumsN  1    1.1695 252.78 -731.40
## - sumsA  1    4.2909 255.90 -722.64
## - sumsC  1    8.7006 260.31 -710.44
## 
## Step:  AIC=-734.67
## hsgpa_num ~ sumsO + sumsN + sumsA + sumsC
## 
##         Df Sum of Sq    RSS     AIC
## - sumsO  1    0.4548 252.07 -735.39
## <none>               251.62 -734.67
## - sumsN  1    1.3502 252.97 -732.85
## - sumsA  1    4.3102 255.93 -724.55
## - sumsC  1    8.8685 260.49 -711.94
## 
## Step:  AIC=-735.39
## hsgpa_num ~ sumsN + sumsA + sumsC
## 
##         Df Sum of Sq    RSS     AIC
## <none>               252.07 -735.39
## - sumsN  1    1.2821 253.36 -733.76
## - sumsA  1    5.9560 258.03 -720.71
## - sumsC  1   10.9783 263.05 -706.95
\end{verbatim}

\begin{verbatim}
## 
## Call:
## lm(formula = hsgpa_num ~ sumsN + sumsA + sumsC, data = dth2hs)
## 
## Coefficients:
## (Intercept)        sumsN        sumsA        sumsC  
##    1.495191    -0.002756     0.007108     0.008022
\end{verbatim}

\begin{verbatim}
## Start:  AIC=-1080.61
## abs ~ sumsA1 + sumsA2 + sumsA3 + sumsA4 + sumsA5 + sumsA6 + sumsA7 + 
##     sumsA8 + sumsC1 + sumsC2 + sumsC3 + sumsC4 + sumsC5 + sumsC6 + 
##     sumsC7 + sumsC8 + sumsC9 + sumsE1 + sumsE2 + sumsE3 + sumsE4 + 
##     sumsE5 + sumsE6 + sumsE7 + sumsE8 + sumsE9 + sumsN1 + sumsN2 + 
##     sumsN3 + sumsN4 + sumsN5 + sumsN6 + sumsN7 + sumsO1 + sumsO2 + 
##     sumsO3 + sumsO4 + sumsO5 + sumsO6 + sumsO7 + sumsO8 + sumsO9
## 
##          Df Sum of Sq    RSS     AIC
## - sumsN2  1   0.00006 143.48 -1082.6
## - sumsC6  1   0.00018 143.48 -1082.6
## - sumsE6  1   0.00079 143.49 -1082.6
## - sumsN6  1   0.00348 143.49 -1082.6
## - sumsA2  1   0.00577 143.49 -1082.6
## - sumsE7  1   0.01118 143.50 -1082.5
## - sumsC4  1   0.01245 143.50 -1082.5
## - sumsO8  1   0.01450 143.50 -1082.5
## - sumsE9  1   0.01838 143.50 -1082.5
## - sumsA6  1   0.04001 143.52 -1082.4
## - sumsC7  1   0.05869 143.54 -1082.3
## - sumsA8  1   0.06326 143.55 -1082.3
## - sumsO7  1   0.06865 143.55 -1082.3
## - sumsA5  1   0.10119 143.59 -1082.1
## - sumsN7  1   0.10763 143.59 -1082.1
## - sumsA4  1   0.11090 143.59 -1082.0
## - sumsN4  1   0.12436 143.61 -1082.0
## - sumsE4  1   0.14497 143.63 -1081.9
## - sumsE3  1   0.15094 143.63 -1081.8
## - sumsE8  1   0.20550 143.69 -1081.6
## - sumsC2  1   0.23583 143.72 -1081.4
## - sumsO9  1   0.26398 143.75 -1081.3
## - sumsO5  1   0.31497 143.80 -1081.0
## - sumsA3  1   0.31793 143.80 -1081.0
## - sumsC5  1   0.32502 143.81 -1081.0
## - sumsE1  1   0.34034 143.82 -1080.9
## - sumsC3  1   0.39262 143.88 -1080.6
## - sumsE5  1   0.39728 143.88 -1080.6
## <none>                143.48 -1080.6
## - sumsC1  1   0.43414 143.92 -1080.4
## - sumsC8  1   0.68333 144.17 -1079.2
## - sumsN5  1   0.77590 144.26 -1078.7
## - sumsA1  1   0.83372 144.32 -1078.4
## - sumsC9  1   0.84470 144.33 -1078.4
## - sumsA7  1   0.90655 144.39 -1078.1
## - sumsO1  1   0.93310 144.42 -1077.9
## - sumsO6  1   0.93415 144.42 -1077.9
## - sumsO4  1   0.96008 144.44 -1077.8
## - sumsO2  1   0.98143 144.47 -1077.7
## - sumsN1  1   1.01074 144.50 -1077.5
## - sumsN3  1   1.04576 144.53 -1077.4
## - sumsE2  1   1.47953 144.96 -1075.2
## - sumsO3  1   1.79248 145.28 -1073.6
## 
## Step:  AIC=-1082.61
## abs ~ sumsA1 + sumsA2 + sumsA3 + sumsA4 + sumsA5 + sumsA6 + sumsA7 + 
##     sumsA8 + sumsC1 + sumsC2 + sumsC3 + sumsC4 + sumsC5 + sumsC6 + 
##     sumsC7 + sumsC8 + sumsC9 + sumsE1 + sumsE2 + sumsE3 + sumsE4 + 
##     sumsE5 + sumsE6 + sumsE7 + sumsE8 + sumsE9 + sumsN1 + sumsN3 + 
##     sumsN4 + sumsN5 + sumsN6 + sumsN7 + sumsO1 + sumsO2 + sumsO3 + 
##     sumsO4 + sumsO5 + sumsO6 + sumsO7 + sumsO8 + sumsO9
## 
##          Df Sum of Sq    RSS     AIC
## - sumsC6  1   0.00019 143.48 -1084.6
## - sumsE6  1   0.00081 143.49 -1084.6
## - sumsN6  1   0.00366 143.49 -1084.6
## - sumsA2  1   0.00581 143.49 -1084.6
## - sumsE7  1   0.01133 143.50 -1084.5
## - sumsC4  1   0.01251 143.50 -1084.5
## - sumsO8  1   0.01462 143.50 -1084.5
## - sumsE9  1   0.01872 143.50 -1084.5
## - sumsA6  1   0.03996 143.52 -1084.4
## - sumsC7  1   0.05870 143.54 -1084.3
## - sumsA8  1   0.06328 143.55 -1084.3
## - sumsO7  1   0.06867 143.55 -1084.3
## - sumsA5  1   0.10123 143.59 -1084.1
## - sumsN7  1   0.10873 143.59 -1084.1
## - sumsA4  1   0.11091 143.59 -1084.0
## - sumsN4  1   0.12447 143.61 -1084.0
## - sumsE3  1   0.15594 143.64 -1083.8
## - sumsE4  1   0.16923 143.65 -1083.8
## - sumsE8  1   0.20713 143.69 -1083.6
## - sumsC2  1   0.23952 143.72 -1083.4
## - sumsO9  1   0.26476 143.75 -1083.3
## - sumsO5  1   0.31769 143.80 -1083.0
## - sumsA3  1   0.31837 143.80 -1083.0
## - sumsC5  1   0.32498 143.81 -1083.0
## - sumsE1  1   0.34137 143.83 -1082.9
## - sumsC3  1   0.39267 143.88 -1082.6
## - sumsE5  1   0.39725 143.88 -1082.6
## <none>                143.48 -1082.6
## - sumsC1  1   0.43409 143.92 -1082.4
## - sumsC8  1   0.68404 144.17 -1081.2
## - sumsN5  1   0.78189 144.27 -1080.7
## - sumsA1  1   0.83460 144.32 -1080.4
## - sumsC9  1   0.84530 144.33 -1080.4
## - sumsA7  1   0.90654 144.39 -1080.1
## - sumsO6  1   0.93543 144.42 -1079.9
## - sumsO1  1   0.94570 144.43 -1079.9
## - sumsO4  1   0.96029 144.44 -1079.8
## - sumsO2  1   0.98266 144.47 -1079.7
## - sumsN1  1   1.01117 144.50 -1079.5
## - sumsN3  1   1.14915 144.63 -1078.8
## - sumsE2  1   1.48252 144.97 -1077.2
## - sumsO3  1   1.80526 145.29 -1075.6
## 
## Step:  AIC=-1084.61
## abs ~ sumsA1 + sumsA2 + sumsA3 + sumsA4 + sumsA5 + sumsA6 + sumsA7 + 
##     sumsA8 + sumsC1 + sumsC2 + sumsC3 + sumsC4 + sumsC5 + sumsC7 + 
##     sumsC8 + sumsC9 + sumsE1 + sumsE2 + sumsE3 + sumsE4 + sumsE5 + 
##     sumsE6 + sumsE7 + sumsE8 + sumsE9 + sumsN1 + sumsN3 + sumsN4 + 
##     sumsN5 + sumsN6 + sumsN7 + sumsO1 + sumsO2 + sumsO3 + sumsO4 + 
##     sumsO5 + sumsO6 + sumsO7 + sumsO8 + sumsO9
## 
##          Df Sum of Sq    RSS     AIC
## - sumsE6  1   0.00073 143.49 -1086.6
## - sumsN6  1   0.00359 143.49 -1086.6
## - sumsA2  1   0.00590 143.49 -1086.6
## - sumsE7  1   0.01147 143.50 -1086.5
## - sumsC4  1   0.01257 143.50 -1086.5
## - sumsO8  1   0.01447 143.50 -1086.5
## - sumsE9  1   0.01889 143.50 -1086.5
## - sumsA6  1   0.03977 143.52 -1086.4
## - sumsC7  1   0.05863 143.54 -1086.3
## - sumsA8  1   0.06327 143.55 -1086.3
## - sumsO7  1   0.06863 143.55 -1086.3
## - sumsA5  1   0.10114 143.59 -1086.1
## - sumsN7  1   0.10976 143.59 -1086.0
## - sumsA4  1   0.11214 143.60 -1086.0
## - sumsN4  1   0.12895 143.61 -1086.0
## - sumsE3  1   0.15577 143.64 -1085.8
## - sumsE4  1   0.16937 143.65 -1085.8
## - sumsE8  1   0.20899 143.69 -1085.6
## - sumsC2  1   0.23937 143.72 -1085.4
## - sumsO9  1   0.26521 143.75 -1085.3
## - sumsA3  1   0.31851 143.80 -1085.0
## - sumsO5  1   0.31861 143.80 -1085.0
## - sumsC5  1   0.32625 143.81 -1085.0
## - sumsE1  1   0.34161 143.83 -1084.9
## - sumsE5  1   0.39776 143.88 -1084.6
## <none>                143.48 -1084.6
## - sumsC3  1   0.40115 143.89 -1084.6
## - sumsC1  1   0.43441 143.92 -1084.4
## - sumsC8  1   0.68392 144.17 -1083.2
## - sumsN5  1   0.78676 144.27 -1082.7
## - sumsA1  1   0.83449 144.32 -1082.4
## - sumsC9  1   0.84512 144.33 -1082.4
## - sumsA7  1   0.91756 144.40 -1082.0
## - sumsO1  1   0.94882 144.43 -1081.8
## - sumsO4  1   0.96022 144.44 -1081.8
## - sumsO6  1   0.98370 144.47 -1081.7
## - sumsO2  1   0.98404 144.47 -1081.7
## - sumsN1  1   1.01138 144.50 -1081.5
## - sumsN3  1   1.15913 144.64 -1080.8
## - sumsE2  1   1.52428 145.01 -1079.0
## - sumsO3  1   1.85075 145.34 -1077.3
## 
## Step:  AIC=-1086.6
## abs ~ sumsA1 + sumsA2 + sumsA3 + sumsA4 + sumsA5 + sumsA6 + sumsA7 + 
##     sumsA8 + sumsC1 + sumsC2 + sumsC3 + sumsC4 + sumsC5 + sumsC7 + 
##     sumsC8 + sumsC9 + sumsE1 + sumsE2 + sumsE3 + sumsE4 + sumsE5 + 
##     sumsE7 + sumsE8 + sumsE9 + sumsN1 + sumsN3 + sumsN4 + sumsN5 + 
##     sumsN6 + sumsN7 + sumsO1 + sumsO2 + sumsO3 + sumsO4 + sumsO5 + 
##     sumsO6 + sumsO7 + sumsO8 + sumsO9
## 
##          Df Sum of Sq    RSS     AIC
## - sumsN6  1   0.00405 143.49 -1088.6
## - sumsA2  1   0.00627 143.49 -1088.6
## - sumsE7  1   0.01206 143.50 -1088.5
## - sumsC4  1   0.01224 143.50 -1088.5
## - sumsO8  1   0.01384 143.50 -1088.5
## - sumsE9  1   0.01841 143.50 -1088.5
## - sumsA6  1   0.04088 143.53 -1088.4
## - sumsC7  1   0.05805 143.54 -1088.3
## - sumsA8  1   0.06284 143.55 -1088.3
## - sumsO7  1   0.06824 143.55 -1088.3
## - sumsA5  1   0.10056 143.59 -1088.1
## - sumsN7  1   0.11128 143.60 -1088.0
## - sumsA4  1   0.11221 143.60 -1088.0
## - sumsN4  1   0.12880 143.61 -1088.0
## - sumsE3  1   0.15843 143.64 -1087.8
## - sumsE4  1   0.16943 143.66 -1087.8
## - sumsE8  1   0.20957 143.69 -1087.5
## - sumsC2  1   0.24255 143.73 -1087.4
## - sumsO9  1   0.26448 143.75 -1087.3
## - sumsO5  1   0.32026 143.81 -1087.0
## - sumsC5  1   0.32809 143.81 -1087.0
## - sumsA3  1   0.33347 143.82 -1086.9
## - sumsE1  1   0.35882 143.84 -1086.8
## <none>                143.49 -1086.6
## - sumsC3  1   0.40305 143.89 -1086.6
## - sumsE5  1   0.40643 143.89 -1086.6
## - sumsC1  1   0.43606 143.92 -1086.4
## - sumsC8  1   0.68422 144.17 -1085.2
## - sumsN5  1   0.78937 144.28 -1084.6
## - sumsA1  1   0.83918 144.32 -1084.4
## - sumsC9  1   0.85256 144.34 -1084.3
## - sumsA7  1   0.92537 144.41 -1084.0
## - sumsO1  1   0.95036 144.44 -1083.8
## - sumsO4  1   0.97587 144.46 -1083.7
## - sumsO6  1   0.98679 144.47 -1083.7
## - sumsO2  1   0.99100 144.48 -1083.6
## - sumsN1  1   1.01077 144.50 -1083.5
## - sumsN3  1   1.15866 144.64 -1082.8
## - sumsE2  1   1.52550 145.01 -1081.0
## - sumsO3  1   1.85031 145.34 -1079.3
## 
## Step:  AIC=-1088.58
## abs ~ sumsA1 + sumsA2 + sumsA3 + sumsA4 + sumsA5 + sumsA6 + sumsA7 + 
##     sumsA8 + sumsC1 + sumsC2 + sumsC3 + sumsC4 + sumsC5 + sumsC7 + 
##     sumsC8 + sumsC9 + sumsE1 + sumsE2 + sumsE3 + sumsE4 + sumsE5 + 
##     sumsE7 + sumsE8 + sumsE9 + sumsN1 + sumsN3 + sumsN4 + sumsN5 + 
##     sumsN7 + sumsO1 + sumsO2 + sumsO3 + sumsO4 + sumsO5 + sumsO6 + 
##     sumsO7 + sumsO8 + sumsO9
## 
##          Df Sum of Sq    RSS     AIC
## - sumsA2  1   0.00486 143.49 -1090.6
## - sumsE7  1   0.01137 143.50 -1090.5
## - sumsC4  1   0.01145 143.50 -1090.5
## - sumsO8  1   0.01186 143.50 -1090.5
## - sumsE9  1   0.01879 143.51 -1090.5
## - sumsA6  1   0.03860 143.53 -1090.4
## - sumsC7  1   0.05742 143.55 -1090.3
## - sumsA8  1   0.06111 143.55 -1090.3
## - sumsO7  1   0.06734 143.56 -1090.2
## - sumsA5  1   0.09799 143.59 -1090.1
## - sumsA4  1   0.11194 143.60 -1090.0
## - sumsN7  1   0.12044 143.61 -1090.0
## - sumsN4  1   0.12479 143.61 -1090.0
## - sumsE3  1   0.15687 143.65 -1089.8
## - sumsE4  1   0.16566 143.66 -1089.8
## - sumsE8  1   0.20860 143.70 -1089.5
## - sumsC2  1   0.24945 143.74 -1089.3
## - sumsO9  1   0.26327 143.75 -1089.3
## - sumsO5  1   0.32056 143.81 -1089.0
## - sumsC5  1   0.32418 143.81 -1089.0
## - sumsA3  1   0.32983 143.82 -1088.9
## - sumsE1  1   0.35993 143.85 -1088.8
## <none>                143.49 -1088.6
## - sumsE5  1   0.40982 143.90 -1088.5
## - sumsC3  1   0.41471 143.90 -1088.5
## - sumsC1  1   0.43698 143.93 -1088.4
## - sumsC8  1   0.68349 144.17 -1087.2
## - sumsN5  1   0.78990 144.28 -1086.6
## - sumsA1  1   0.83536 144.32 -1086.4
## - sumsC9  1   0.84901 144.34 -1086.3
## - sumsA7  1   0.92159 144.41 -1086.0
## - sumsO1  1   0.95114 144.44 -1085.8
## - sumsO4  1   0.98058 144.47 -1085.7
## - sumsO6  1   0.98325 144.47 -1085.7
## - sumsO2  1   0.99236 144.48 -1085.6
## - sumsN1  1   1.00807 144.50 -1085.5
## - sumsN3  1   1.26758 144.76 -1084.2
## - sumsE2  1   1.52914 145.02 -1082.9
## - sumsO3  1   1.84694 145.34 -1081.3
## 
## Step:  AIC=-1090.56
## abs ~ sumsA1 + sumsA3 + sumsA4 + sumsA5 + sumsA6 + sumsA7 + sumsA8 + 
##     sumsC1 + sumsC2 + sumsC3 + sumsC4 + sumsC5 + sumsC7 + sumsC8 + 
##     sumsC9 + sumsE1 + sumsE2 + sumsE3 + sumsE4 + sumsE5 + sumsE7 + 
##     sumsE8 + sumsE9 + sumsN1 + sumsN3 + sumsN4 + sumsN5 + sumsN7 + 
##     sumsO1 + sumsO2 + sumsO3 + sumsO4 + sumsO5 + sumsO6 + sumsO7 + 
##     sumsO8 + sumsO9
## 
##          Df Sum of Sq    RSS     AIC
## - sumsE7  1   0.01060 143.50 -1092.5
## - sumsC4  1   0.01113 143.50 -1092.5
## - sumsO8  1   0.01259 143.51 -1092.5
## - sumsE9  1   0.01716 143.51 -1092.5
## - sumsA6  1   0.03603 143.53 -1092.4
## - sumsC7  1   0.05514 143.55 -1092.3
## - sumsO7  1   0.06509 143.56 -1092.2
## - sumsA8  1   0.06695 143.56 -1092.2
## - sumsA5  1   0.09480 143.59 -1092.1
## - sumsA4  1   0.11666 143.61 -1092.0
## - sumsN7  1   0.12272 143.62 -1091.9
## - sumsN4  1   0.12671 143.62 -1091.9
## - sumsE3  1   0.15805 143.65 -1091.8
## - sumsE4  1   0.16347 143.66 -1091.7
## - sumsE8  1   0.20999 143.70 -1091.5
## - sumsC2  1   0.25686 143.75 -1091.3
## - sumsO9  1   0.26170 143.76 -1091.2
## - sumsO5  1   0.32309 143.82 -1090.9
## - sumsC5  1   0.32818 143.82 -1090.9
## - sumsA3  1   0.35190 143.85 -1090.8
## - sumsE1  1   0.35737 143.85 -1090.8
## <none>                143.49 -1090.6
## - sumsE5  1   0.41142 143.91 -1090.5
## - sumsC3  1   0.41285 143.91 -1090.5
## - sumsC1  1   0.43907 143.93 -1090.3
## - sumsC8  1   0.68798 144.18 -1089.1
## - sumsN5  1   0.78820 144.28 -1088.6
## - sumsC9  1   0.84463 144.34 -1088.3
## - sumsA1  1   0.86536 144.36 -1088.2
## - sumsA7  1   0.94436 144.44 -1087.8
## - sumsO1  1   0.94659 144.44 -1087.8
## - sumsO6  1   0.97910 144.47 -1087.7
## - sumsO4  1   0.99297 144.49 -1087.6
## - sumsO2  1   0.99323 144.49 -1087.6
## - sumsN1  1   1.02120 144.51 -1087.4
## - sumsN3  1   1.26316 144.76 -1086.2
## - sumsE2  1   1.53058 145.03 -1084.9
## - sumsO3  1   1.85645 145.35 -1083.3
## 
## Step:  AIC=-1092.5
## abs ~ sumsA1 + sumsA3 + sumsA4 + sumsA5 + sumsA6 + sumsA7 + sumsA8 + 
##     sumsC1 + sumsC2 + sumsC3 + sumsC4 + sumsC5 + sumsC7 + sumsC8 + 
##     sumsC9 + sumsE1 + sumsE2 + sumsE3 + sumsE4 + sumsE5 + sumsE8 + 
##     sumsE9 + sumsN1 + sumsN3 + sumsN4 + sumsN5 + sumsN7 + sumsO1 + 
##     sumsO2 + sumsO3 + sumsO4 + sumsO5 + sumsO6 + sumsO7 + sumsO8 + 
##     sumsO9
## 
##          Df Sum of Sq    RSS     AIC
## - sumsC4  1   0.00956 143.51 -1094.5
## - sumsO8  1   0.01306 143.52 -1094.4
## - sumsE9  1   0.01667 143.52 -1094.4
## - sumsA6  1   0.03365 143.54 -1094.3
## - sumsC7  1   0.05594 143.56 -1094.2
## - sumsO7  1   0.06473 143.57 -1094.2
## - sumsA8  1   0.06742 143.57 -1094.2
## - sumsA5  1   0.09563 143.60 -1094.0
## - sumsA4  1   0.11503 143.62 -1093.9
## - sumsN7  1   0.12283 143.63 -1093.9
## - sumsN4  1   0.12551 143.63 -1093.9
## - sumsE4  1   0.15318 143.66 -1093.7
## - sumsE3  1   0.16286 143.67 -1093.7
## - sumsE8  1   0.21472 143.72 -1093.4
## - sumsC2  1   0.25704 143.76 -1093.2
## - sumsO9  1   0.25883 143.76 -1093.2
## - sumsO5  1   0.32319 143.83 -1092.9
## - sumsC5  1   0.33566 143.84 -1092.8
## - sumsA3  1   0.35303 143.86 -1092.7
## - sumsE1  1   0.37530 143.88 -1092.6
## <none>                143.50 -1092.5
## - sumsE5  1   0.40718 143.91 -1092.5
## - sumsC1  1   0.43376 143.94 -1092.3
## - sumsC3  1   0.44293 143.95 -1092.3
## - sumsC8  1   0.71336 144.22 -1090.9
## - sumsN5  1   0.77892 144.28 -1090.6
## - sumsC9  1   0.84271 144.35 -1090.3
## - sumsA1  1   0.85516 144.36 -1090.2
## - sumsA7  1   0.93396 144.44 -1089.8
## - sumsO1  1   0.93664 144.44 -1089.8
## - sumsO6  1   0.99322 144.50 -1089.5
## - sumsO2  1   0.99843 144.50 -1089.5
## - sumsO4  1   1.01236 144.52 -1089.4
## - sumsN1  1   1.02401 144.53 -1089.4
## - sumsN3  1   1.25257 144.76 -1088.2
## - sumsE2  1   1.52220 145.03 -1086.9
## - sumsO3  1   1.90304 145.41 -1085.0
## 
## Step:  AIC=-1094.46
## abs ~ sumsA1 + sumsA3 + sumsA4 + sumsA5 + sumsA6 + sumsA7 + sumsA8 + 
##     sumsC1 + sumsC2 + sumsC3 + sumsC5 + sumsC7 + sumsC8 + sumsC9 + 
##     sumsE1 + sumsE2 + sumsE3 + sumsE4 + sumsE5 + sumsE8 + sumsE9 + 
##     sumsN1 + sumsN3 + sumsN4 + sumsN5 + sumsN7 + sumsO1 + sumsO2 + 
##     sumsO3 + sumsO4 + sumsO5 + sumsO6 + sumsO7 + sumsO8 + sumsO9
## 
##          Df Sum of Sq    RSS     AIC
## - sumsO8  1   0.01270 143.53 -1096.4
## - sumsE9  1   0.01692 143.53 -1096.4
## - sumsA6  1   0.03779 143.55 -1096.3
## - sumsC7  1   0.04994 143.56 -1096.2
## - sumsO7  1   0.06584 143.58 -1096.1
## - sumsA8  1   0.06775 143.58 -1096.1
## - sumsA5  1   0.09763 143.61 -1096.0
## - sumsA4  1   0.10805 143.62 -1095.9
## - sumsN7  1   0.12388 143.64 -1095.8
## - sumsN4  1   0.12512 143.64 -1095.8
## - sumsE4  1   0.15408 143.67 -1095.7
## - sumsE3  1   0.17399 143.69 -1095.6
## - sumsE8  1   0.20564 143.72 -1095.4
## - sumsC2  1   0.25742 143.77 -1095.2
## - sumsO9  1   0.25871 143.77 -1095.2
## - sumsO5  1   0.31582 143.83 -1094.9
## - sumsA3  1   0.34713 143.86 -1094.7
## - sumsE1  1   0.37686 143.89 -1094.6
## <none>                143.51 -1094.5
## - sumsE5  1   0.40619 143.92 -1094.4
## - sumsC1  1   0.42806 143.94 -1094.3
## - sumsC5  1   0.42809 143.94 -1094.3
## - sumsC3  1   0.46924 143.98 -1094.1
## - sumsC8  1   0.75090 144.26 -1092.7
## - sumsN5  1   0.78321 144.30 -1092.5
## - sumsA1  1   0.85953 144.37 -1092.2
## - sumsC9  1   0.89067 144.41 -1092.0
## - sumsA7  1   0.92806 144.44 -1091.8
## - sumsO1  1   0.93252 144.45 -1091.8
## - sumsO6  1   0.98376 144.50 -1091.5
## - sumsO2  1   0.99535 144.51 -1091.5
## - sumsO4  1   1.01753 144.53 -1091.4
## - sumsN1  1   1.07334 144.59 -1091.1
## - sumsN3  1   1.24462 144.76 -1090.2
## - sumsE2  1   1.52534 145.04 -1088.8
## - sumsO3  1   1.90467 145.42 -1086.9
## 
## Step:  AIC=-1096.39
## abs ~ sumsA1 + sumsA3 + sumsA4 + sumsA5 + sumsA6 + sumsA7 + sumsA8 + 
##     sumsC1 + sumsC2 + sumsC3 + sumsC5 + sumsC7 + sumsC8 + sumsC9 + 
##     sumsE1 + sumsE2 + sumsE3 + sumsE4 + sumsE5 + sumsE8 + sumsE9 + 
##     sumsN1 + sumsN3 + sumsN4 + sumsN5 + sumsN7 + sumsO1 + sumsO2 + 
##     sumsO3 + sumsO4 + sumsO5 + sumsO6 + sumsO7 + sumsO9
## 
##          Df Sum of Sq    RSS     AIC
## - sumsE9  1   0.01863 143.55 -1098.3
## - sumsA6  1   0.03172 143.56 -1098.2
## - sumsC7  1   0.05034 143.58 -1098.1
## - sumsO7  1   0.06309 143.59 -1098.1
## - sumsA8  1   0.06715 143.59 -1098.1
## - sumsA4  1   0.11017 143.64 -1097.8
## - sumsN7  1   0.12204 143.65 -1097.8
## - sumsA5  1   0.12807 143.66 -1097.8
## - sumsN4  1   0.13348 143.66 -1097.7
## - sumsE4  1   0.15813 143.69 -1097.6
## - sumsE3  1   0.16503 143.69 -1097.6
## - sumsE8  1   0.21166 143.74 -1097.3
## - sumsC2  1   0.25321 143.78 -1097.1
## - sumsO9  1   0.25339 143.78 -1097.1
## - sumsO5  1   0.32268 143.85 -1096.8
## - sumsA3  1   0.34901 143.88 -1096.6
## - sumsE1  1   0.36718 143.89 -1096.5
## <none>                143.53 -1096.4
## - sumsE5  1   0.40647 143.93 -1096.3
## - sumsC5  1   0.42467 143.95 -1096.3
## - sumsC1  1   0.43157 143.96 -1096.2
## - sumsC3  1   0.47270 144.00 -1096.0
## - sumsC8  1   0.75292 144.28 -1094.6
## - sumsN5  1   0.77854 144.31 -1094.5
## - sumsC9  1   0.87941 144.41 -1094.0
## - sumsA1  1   0.90031 144.43 -1093.9
## - sumsA7  1   0.91537 144.44 -1093.8
## - sumsO1  1   0.94640 144.47 -1093.7
## - sumsO6  1   0.97106 144.50 -1093.5
## - sumsO2  1   0.98787 144.51 -1093.4
## - sumsO4  1   1.02577 144.55 -1093.2
## - sumsN1  1   1.07937 144.61 -1093.0
## - sumsN3  1   1.24927 144.78 -1092.1
## - sumsE2  1   1.51743 145.04 -1090.8
## - sumsO3  1   1.92603 145.45 -1088.8
## 
## Step:  AIC=-1098.3
## abs ~ sumsA1 + sumsA3 + sumsA4 + sumsA5 + sumsA6 + sumsA7 + sumsA8 + 
##     sumsC1 + sumsC2 + sumsC3 + sumsC5 + sumsC7 + sumsC8 + sumsC9 + 
##     sumsE1 + sumsE2 + sumsE3 + sumsE4 + sumsE5 + sumsE8 + sumsN1 + 
##     sumsN3 + sumsN4 + sumsN5 + sumsN7 + sumsO1 + sumsO2 + sumsO3 + 
##     sumsO4 + sumsO5 + sumsO6 + sumsO7 + sumsO9
## 
##          Df Sum of Sq    RSS     AIC
## - sumsA6  1   0.03138 143.58 -1100.1
## - sumsC7  1   0.04759 143.59 -1100.1
## - sumsO7  1   0.06656 143.61 -1100.0
## - sumsA8  1   0.07286 143.62 -1099.9
## - sumsA4  1   0.10923 143.66 -1099.8
## - sumsA5  1   0.11806 143.66 -1099.7
## - sumsN7  1   0.11903 143.66 -1099.7
## - sumsN4  1   0.13052 143.68 -1099.6
## - sumsE3  1   0.15362 143.70 -1099.5
## - sumsE4  1   0.17174 143.72 -1099.4
## - sumsE8  1   0.20074 143.75 -1099.3
## - sumsO9  1   0.25498 143.80 -1099.0
## - sumsC2  1   0.26140 143.81 -1099.0
## - sumsO5  1   0.31868 143.86 -1098.7
## - sumsA3  1   0.35028 143.90 -1098.5
## - sumsE1  1   0.35979 143.91 -1098.5
## <none>                143.55 -1098.3
## - sumsE5  1   0.39903 143.94 -1098.3
## - sumsC1  1   0.42553 143.97 -1098.2
## - sumsC5  1   0.44127 143.99 -1098.1
## - sumsC3  1   0.47871 144.02 -1097.9
## - sumsC8  1   0.76269 144.31 -1096.5
## - sumsN5  1   0.77320 144.32 -1096.4
## - sumsA1  1   0.88964 144.44 -1095.8
## - sumsA7  1   0.91814 144.46 -1095.7
## - sumsC9  1   0.92778 144.47 -1095.7
## - sumsO6  1   0.97928 144.53 -1095.4
## - sumsO1  1   0.98274 144.53 -1095.4
## - sumsO2  1   0.99734 144.54 -1095.3
## - sumsO4  1   1.01549 144.56 -1095.2
## - sumsN1  1   1.07050 144.62 -1094.9
## - sumsN3  1   1.27297 144.82 -1093.9
## - sumsE2  1   1.59869 145.14 -1092.3
## - sumsO3  1   1.91370 145.46 -1090.7
## 
## Step:  AIC=-1100.14
## abs ~ sumsA1 + sumsA3 + sumsA4 + sumsA5 + sumsA7 + sumsA8 + sumsC1 + 
##     sumsC2 + sumsC3 + sumsC5 + sumsC7 + sumsC8 + sumsC9 + sumsE1 + 
##     sumsE2 + sumsE3 + sumsE4 + sumsE5 + sumsE8 + sumsN1 + sumsN3 + 
##     sumsN4 + sumsN5 + sumsN7 + sumsO1 + sumsO2 + sumsO3 + sumsO4 + 
##     sumsO5 + sumsO6 + sumsO7 + sumsO9
## 
##          Df Sum of Sq    RSS     AIC
## - sumsC7  1   0.04821 143.62 -1101.9
## - sumsO7  1   0.06301 143.64 -1101.8
## - sumsA8  1   0.06334 143.64 -1101.8
## - sumsA5  1   0.10835 143.69 -1101.6
## - sumsN7  1   0.11473 143.69 -1101.6
## - sumsN4  1   0.12458 143.70 -1101.5
## - sumsA4  1   0.13768 143.72 -1101.5
## - sumsE4  1   0.14859 143.73 -1101.4
## - sumsE3  1   0.15472 143.73 -1101.4
## - sumsE8  1   0.24605 143.82 -1100.9
## - sumsO9  1   0.26021 143.84 -1100.8
## - sumsC2  1   0.26543 143.84 -1100.8
## - sumsO5  1   0.33001 143.91 -1100.5
## - sumsA3  1   0.33923 143.92 -1100.4
## - sumsE1  1   0.37300 143.95 -1100.3
## <none>                143.58 -1100.1
## - sumsE5  1   0.40670 143.98 -1100.1
## - sumsC1  1   0.41776 144.00 -1100.0
## - sumsC5  1   0.42147 144.00 -1100.0
## - sumsC3  1   0.50713 144.08 -1099.6
## - sumsN5  1   0.75884 144.34 -1098.3
## - sumsC8  1   0.78652 144.36 -1098.2
## - sumsA1  1   0.86335 144.44 -1097.8
## - sumsA7  1   0.91318 144.49 -1097.6
## - sumsC9  1   0.91321 144.49 -1097.6
## - sumsO6  1   0.96402 144.54 -1097.3
## - sumsO1  1   0.98733 144.56 -1097.2
## - sumsO2  1   1.01547 144.59 -1097.0
## - sumsO4  1   1.03346 144.61 -1097.0
## - sumsN1  1   1.15778 144.74 -1096.3
## - sumsN3  1   1.25167 144.83 -1095.9
## - sumsE2  1   1.64527 145.22 -1093.9
## - sumsO3  1   1.93899 145.52 -1092.5
## 
## Step:  AIC=-1101.9
## abs ~ sumsA1 + sumsA3 + sumsA4 + sumsA5 + sumsA7 + sumsA8 + sumsC1 + 
##     sumsC2 + sumsC3 + sumsC5 + sumsC8 + sumsC9 + sumsE1 + sumsE2 + 
##     sumsE3 + sumsE4 + sumsE5 + sumsE8 + sumsN1 + sumsN3 + sumsN4 + 
##     sumsN5 + sumsN7 + sumsO1 + sumsO2 + sumsO3 + sumsO4 + sumsO5 + 
##     sumsO6 + sumsO7 + sumsO9
## 
##          Df Sum of Sq    RSS     AIC
## - sumsO7  1   0.06243 143.69 -1103.6
## - sumsA8  1   0.08082 143.71 -1103.5
## - sumsA5  1   0.09938 143.72 -1103.4
## - sumsN7  1   0.11291 143.74 -1103.3
## - sumsN4  1   0.12630 143.75 -1103.3
## - sumsA4  1   0.15268 143.78 -1103.1
## - sumsE4  1   0.15845 143.78 -1103.1
## - sumsE3  1   0.16394 143.79 -1103.1
## - sumsC2  1   0.25390 143.88 -1102.6
## - sumsE8  1   0.25669 143.88 -1102.6
## - sumsO9  1   0.27563 143.90 -1102.5
## - sumsA3  1   0.32856 143.95 -1102.2
## - sumsO5  1   0.33571 143.96 -1102.2
## - sumsE1  1   0.37269 144.00 -1102.0
## - sumsC5  1   0.38691 144.01 -1102.0
## <none>                143.62 -1101.9
## - sumsC1  1   0.42618 144.05 -1101.8
## - sumsE5  1   0.42835 144.05 -1101.8
## - sumsC3  1   0.46980 144.09 -1101.5
## - sumsN5  1   0.77355 144.40 -1100.0
## - sumsC8  1   0.82566 144.45 -1099.8
## - sumsA1  1   0.86390 144.49 -1099.6
## - sumsC9  1   0.87418 144.50 -1099.5
## - sumsA7  1   0.94287 144.57 -1099.2
## - sumsO6  1   0.96536 144.59 -1099.1
## - sumsO1  1   0.97366 144.60 -1099.0
## - sumsO4  1   0.99772 144.62 -1098.9
## - sumsO2  1   1.00163 144.63 -1098.9
## - sumsN1  1   1.16609 144.79 -1098.1
## - sumsN3  1   1.21314 144.84 -1097.8
## - sumsE2  1   1.69311 145.32 -1095.4
## - sumsO3  1   1.93312 145.56 -1094.2
## 
## Step:  AIC=-1103.59
## abs ~ sumsA1 + sumsA3 + sumsA4 + sumsA5 + sumsA7 + sumsA8 + sumsC1 + 
##     sumsC2 + sumsC3 + sumsC5 + sumsC8 + sumsC9 + sumsE1 + sumsE2 + 
##     sumsE3 + sumsE4 + sumsE5 + sumsE8 + sumsN1 + sumsN3 + sumsN4 + 
##     sumsN5 + sumsN7 + sumsO1 + sumsO2 + sumsO3 + sumsO4 + sumsO5 + 
##     sumsO6 + sumsO9
## 
##          Df Sum of Sq    RSS     AIC
## - sumsA8  1   0.07930 143.77 -1105.2
## - sumsA5  1   0.11425 143.80 -1105.0
## - sumsN4  1   0.12243 143.81 -1105.0
## - sumsN7  1   0.12975 143.82 -1104.9
## - sumsE4  1   0.13429 143.82 -1104.9
## - sumsA4  1   0.13792 143.83 -1104.9
## - sumsE3  1   0.15451 143.84 -1104.8
## - sumsE8  1   0.25100 143.94 -1104.3
## - sumsC2  1   0.25162 143.94 -1104.3
## - sumsO9  1   0.28460 143.97 -1104.2
## - sumsO5  1   0.31445 144.00 -1104.0
## - sumsA3  1   0.32539 144.01 -1104.0
## - sumsC5  1   0.35797 144.05 -1103.8
## - sumsE1  1   0.38887 144.08 -1103.6
## <none>                143.69 -1103.6
## - sumsC1  1   0.42063 144.11 -1103.5
## - sumsE5  1   0.43746 144.12 -1103.4
## - sumsC3  1   0.45797 144.15 -1103.3
## - sumsC8  1   0.80219 144.49 -1101.6
## - sumsA1  1   0.80482 144.49 -1101.5
## - sumsN5  1   0.81490 144.50 -1101.5
## - sumsO1  1   0.92298 144.61 -1101.0
## - sumsC9  1   0.93278 144.62 -1100.9
## - sumsA7  1   0.98636 144.67 -1100.7
## - sumsO4  1   1.12050 144.81 -1100.0
## - sumsN1  1   1.16323 144.85 -1099.8
## - sumsO2  1   1.16369 144.85 -1099.8
## - sumsO6  1   1.18221 144.87 -1099.7
## - sumsN3  1   1.23109 144.92 -1099.4
## - sumsE2  1   1.74489 145.43 -1096.9
## - sumsO3  1   1.93792 145.63 -1095.9
## 
## Step:  AIC=-1105.19
## abs ~ sumsA1 + sumsA3 + sumsA4 + sumsA5 + sumsA7 + sumsC1 + sumsC2 + 
##     sumsC3 + sumsC5 + sumsC8 + sumsC9 + sumsE1 + sumsE2 + sumsE3 + 
##     sumsE4 + sumsE5 + sumsE8 + sumsN1 + sumsN3 + sumsN4 + sumsN5 + 
##     sumsN7 + sumsO1 + sumsO2 + sumsO3 + sumsO4 + sumsO5 + sumsO6 + 
##     sumsO9
## 
##          Df Sum of Sq    RSS     AIC
## - sumsN4  1   0.10880 143.88 -1106.6
## - sumsA4  1   0.12272 143.89 -1106.6
## - sumsN7  1   0.13093 143.90 -1106.5
## - sumsA5  1   0.13484 143.90 -1106.5
## - sumsE4  1   0.14738 143.91 -1106.5
## - sumsE3  1   0.16495 143.93 -1106.4
## - sumsE8  1   0.22993 144.00 -1106.0
## - sumsC2  1   0.25413 144.02 -1105.9
## - sumsO9  1   0.28562 144.05 -1105.8
## - sumsO5  1   0.33918 144.11 -1105.5
## - sumsC5  1   0.36025 144.13 -1105.4
## - sumsA3  1   0.38198 144.15 -1105.3
## <none>                143.77 -1105.2
## - sumsC1  1   0.40794 144.18 -1105.1
## - sumsE1  1   0.41081 144.18 -1105.1
## - sumsC3  1   0.44878 144.22 -1104.9
## - sumsE5  1   0.46813 144.24 -1104.8
## - sumsC8  1   0.85324 144.62 -1102.9
## - sumsN5  1   0.87799 144.65 -1102.8
## - sumsO1  1   0.92702 144.69 -1102.5
## - sumsA7  1   0.93792 144.71 -1102.5
## - sumsC9  1   0.99631 144.76 -1102.2
## - sumsA1  1   1.00878 144.78 -1102.1
## - sumsN1  1   1.09410 144.86 -1101.7
## - sumsO4  1   1.10671 144.87 -1101.7
## - sumsO6  1   1.20308 144.97 -1101.2
## - sumsN3  1   1.21143 144.98 -1101.1
## - sumsO2  1   1.23645 145.00 -1101.0
## - sumsE2  1   1.74913 145.52 -1098.5
## - sumsO3  1   1.93548 145.70 -1097.5
## 
## Step:  AIC=-1106.64
## abs ~ sumsA1 + sumsA3 + sumsA4 + sumsA5 + sumsA7 + sumsC1 + sumsC2 + 
##     sumsC3 + sumsC5 + sumsC8 + sumsC9 + sumsE1 + sumsE2 + sumsE3 + 
##     sumsE4 + sumsE5 + sumsE8 + sumsN1 + sumsN3 + sumsN5 + sumsN7 + 
##     sumsO1 + sumsO2 + sumsO3 + sumsO4 + sumsO5 + sumsO6 + sumsO9
## 
##          Df Sum of Sq    RSS     AIC
## - sumsE4  1   0.11804 143.99 -1108.0
## - sumsA4  1   0.11946 144.00 -1108.0
## - sumsN7  1   0.12563 144.00 -1108.0
## - sumsE3  1   0.12808 144.00 -1108.0
## - sumsA5  1   0.16854 144.04 -1107.8
## - sumsC2  1   0.22283 144.10 -1107.5
## - sumsE8  1   0.24530 144.12 -1107.4
## - sumsO9  1   0.34611 144.22 -1106.9
## - sumsO5  1   0.35119 144.23 -1106.9
## - sumsC5  1   0.35389 144.23 -1106.9
## <none>                143.88 -1106.6
## - sumsC1  1   0.41822 144.29 -1106.5
## - sumsA3  1   0.42963 144.31 -1106.5
## - sumsE1  1   0.44147 144.32 -1106.4
## - sumsC3  1   0.45679 144.33 -1106.3
## - sumsE5  1   0.46750 144.34 -1106.3
## - sumsC8  1   0.83399 144.71 -1104.5
## - sumsN5  1   0.85595 144.73 -1104.4
## - sumsO1  1   0.86601 144.74 -1104.3
## - sumsC9  1   0.94427 144.82 -1103.9
## - sumsA7  1   0.96706 144.84 -1103.8
## - sumsA1  1   1.03415 144.91 -1103.5
## - sumsO4  1   1.09891 144.97 -1103.2
## - sumsN3  1   1.13500 145.01 -1103.0
## - sumsN1  1   1.15898 145.03 -1102.8
## - sumsO6  1   1.24368 145.12 -1102.4
## - sumsO2  1   1.25085 145.13 -1102.4
## - sumsE2  1   1.66349 145.54 -1100.3
## - sumsO3  1   2.16044 146.04 -1097.9
## 
## Step:  AIC=-1108.05
## abs ~ sumsA1 + sumsA3 + sumsA4 + sumsA5 + sumsA7 + sumsC1 + sumsC2 + 
##     sumsC3 + sumsC5 + sumsC8 + sumsC9 + sumsE1 + sumsE2 + sumsE3 + 
##     sumsE5 + sumsE8 + sumsN1 + sumsN3 + sumsN5 + sumsN7 + sumsO1 + 
##     sumsO2 + sumsO3 + sumsO4 + sumsO5 + sumsO6 + sumsO9
## 
##          Df Sum of Sq    RSS     AIC
## - sumsA4  1   0.10435 144.10 -1109.5
## - sumsE3  1   0.11724 144.11 -1109.5
## - sumsN7  1   0.11767 144.11 -1109.5
## - sumsA5  1   0.18264 144.18 -1109.1
## - sumsE8  1   0.19589 144.19 -1109.1
## - sumsC2  1   0.24407 144.24 -1108.8
## - sumsO9  1   0.33283 144.33 -1108.4
## - sumsO5  1   0.35530 144.35 -1108.3
## <none>                143.99 -1108.0
## - sumsC3  1   0.40855 144.40 -1108.0
## - sumsC5  1   0.41118 144.41 -1108.0
## - sumsA3  1   0.41912 144.41 -1108.0
## - sumsE1  1   0.42098 144.41 -1107.9
## - sumsC1  1   0.42694 144.42 -1107.9
## - sumsE5  1   0.47094 144.47 -1107.7
## - sumsN5  1   0.82395 144.82 -1105.9
## - sumsC8  1   0.84022 144.83 -1105.8
## - sumsO1  1   0.91986 144.91 -1105.5
## - sumsA7  1   0.93294 144.93 -1105.4
## - sumsC9  1   1.00023 144.99 -1105.0
## - sumsO4  1   1.11021 145.10 -1104.5
## - sumsN1  1   1.12114 145.12 -1104.5
## - sumsA1  1   1.12474 145.12 -1104.4
## - sumsO2  1   1.17045 145.16 -1104.2
## - sumsO6  1   1.22599 145.22 -1103.9
## - sumsN3  1   1.39784 145.39 -1103.1
## - sumsE2  1   1.66922 145.66 -1101.7
## - sumsO3  1   2.13345 146.13 -1099.4
## 
## Step:  AIC=-1109.53
## abs ~ sumsA1 + sumsA3 + sumsA5 + sumsA7 + sumsC1 + sumsC2 + sumsC3 + 
##     sumsC5 + sumsC8 + sumsC9 + sumsE1 + sumsE2 + sumsE3 + sumsE5 + 
##     sumsE8 + sumsN1 + sumsN3 + sumsN5 + sumsN7 + sumsO1 + sumsO2 + 
##     sumsO3 + sumsO4 + sumsO5 + sumsO6 + sumsO9
## 
##          Df Sum of Sq    RSS     AIC
## - sumsE3  1   0.10856 144.21 -1111.0
## - sumsN7  1   0.20024 144.30 -1110.5
## - sumsA5  1   0.20335 144.30 -1110.5
## - sumsE8  1   0.25153 144.35 -1110.3
## - sumsC2  1   0.26788 144.37 -1110.2
## - sumsO9  1   0.31346 144.41 -1110.0
## - sumsO5  1   0.39387 144.49 -1109.5
## <none>                144.10 -1109.5
## - sumsC5  1   0.41016 144.51 -1109.5
## - sumsC1  1   0.42813 144.53 -1109.4
## - sumsE1  1   0.43096 144.53 -1109.4
## - sumsC3  1   0.43835 144.54 -1109.3
## - sumsA3  1   0.50582 144.60 -1109.0
## - sumsE5  1   0.52172 144.62 -1108.9
## - sumsN5  1   0.77895 144.88 -1107.6
## - sumsC8  1   0.80727 144.91 -1107.5
## - sumsA7  1   0.93729 145.03 -1106.8
## - sumsC9  1   0.94359 145.04 -1106.8
## - sumsO1  1   1.00905 145.11 -1106.5
## - sumsA1  1   1.04466 145.14 -1106.3
## - sumsO4  1   1.11874 145.22 -1105.9
## - sumsO2  1   1.18305 145.28 -1105.6
## - sumsO6  1   1.19235 145.29 -1105.6
## - sumsN1  1   1.25215 145.35 -1105.3
## - sumsE2  1   1.65736 145.75 -1103.3
## - sumsN3  1   1.66132 145.76 -1103.2
## - sumsO3  1   2.14466 146.24 -1100.9
## 
## Step:  AIC=-1110.98
## abs ~ sumsA1 + sumsA3 + sumsA5 + sumsA7 + sumsC1 + sumsC2 + sumsC3 + 
##     sumsC5 + sumsC8 + sumsC9 + sumsE1 + sumsE2 + sumsE5 + sumsE8 + 
##     sumsN1 + sumsN3 + sumsN5 + sumsN7 + sumsO1 + sumsO2 + sumsO3 + 
##     sumsO4 + sumsO5 + sumsO6 + sumsO9
## 
##          Df Sum of Sq    RSS     AIC
## - sumsA5  1   0.17933 144.39 -1112.1
## - sumsN7  1   0.18387 144.39 -1112.1
## - sumsC2  1   0.22456 144.43 -1111.9
## - sumsO9  1   0.28454 144.49 -1111.6
## - sumsE8  1   0.33544 144.54 -1111.3
## <none>                144.21 -1111.0
## - sumsC1  1   0.40117 144.61 -1111.0
## - sumsO5  1   0.41558 144.62 -1110.9
## - sumsC5  1   0.41611 144.62 -1110.9
## - sumsC3  1   0.46238 144.67 -1110.7
## - sumsE5  1   0.49869 144.71 -1110.5
## - sumsA3  1   0.54616 144.75 -1110.2
## - sumsE1  1   0.60495 144.81 -1110.0
## - sumsN5  1   0.80772 145.01 -1109.0
## - sumsC8  1   0.81410 145.02 -1108.9
## - sumsC9  1   0.97421 145.18 -1108.1
## - sumsA7  1   1.01321 145.22 -1107.9
## - sumsA1  1   1.05064 145.26 -1107.7
## - sumsO4  1   1.11920 145.33 -1107.4
## - sumsO1  1   1.12669 145.33 -1107.4
## - sumsO6  1   1.14783 145.35 -1107.3
## - sumsO2  1   1.18253 145.39 -1107.1
## - sumsN1  1   1.31541 145.52 -1106.4
## - sumsN3  1   1.59598 145.80 -1105.0
## - sumsE2  1   1.61777 145.82 -1104.9
## - sumsO3  1   2.15776 146.36 -1102.3
## 
## Step:  AIC=-1112.08
## abs ~ sumsA1 + sumsA3 + sumsA7 + sumsC1 + sumsC2 + sumsC3 + sumsC5 + 
##     sumsC8 + sumsC9 + sumsE1 + sumsE2 + sumsE5 + sumsE8 + sumsN1 + 
##     sumsN3 + sumsN5 + sumsN7 + sumsO1 + sumsO2 + sumsO3 + sumsO4 + 
##     sumsO5 + sumsO6 + sumsO9
## 
##          Df Sum of Sq    RSS     AIC
## - sumsN7  1   0.15363 144.54 -1113.3
## - sumsC2  1   0.17847 144.56 -1113.2
## - sumsE8  1   0.30108 144.69 -1112.6
## - sumsO9  1   0.34658 144.73 -1112.3
## - sumsC3  1   0.38916 144.78 -1112.1
## - sumsC1  1   0.39178 144.78 -1112.1
## <none>                144.39 -1112.1
## - sumsC5  1   0.43992 144.83 -1111.9
## - sumsO5  1   0.45941 144.84 -1111.8
## - sumsE5  1   0.50304 144.89 -1111.6
## - sumsE1  1   0.51869 144.91 -1111.5
## - sumsA3  1   0.52369 144.91 -1111.5
## - sumsC8  1   0.77317 145.16 -1110.2
## - sumsN5  1   0.84057 145.23 -1109.9
## - sumsA7  1   0.91733 145.30 -1109.5
## - sumsC9  1   0.96064 145.35 -1109.3
## - sumsA1  1   1.07991 145.47 -1108.7
## - sumsO1  1   1.09645 145.48 -1108.6
## - sumsO4  1   1.10340 145.49 -1108.6
## - sumsO6  1   1.12032 145.51 -1108.5
## - sumsO2  1   1.22278 145.61 -1108.0
## - sumsN3  1   1.45264 145.84 -1106.9
## - sumsE2  1   1.52938 145.91 -1106.5
## - sumsN1  1   1.78818 146.17 -1105.2
## - sumsO3  1   2.18043 146.57 -1103.3
## 
## Step:  AIC=-1113.32
## abs ~ sumsA1 + sumsA3 + sumsA7 + sumsC1 + sumsC2 + sumsC3 + sumsC5 + 
##     sumsC8 + sumsC9 + sumsE1 + sumsE2 + sumsE5 + sumsE8 + sumsN1 + 
##     sumsN3 + sumsN5 + sumsO1 + sumsO2 + sumsO3 + sumsO4 + sumsO5 + 
##     sumsO6 + sumsO9
## 
##          Df Sum of Sq    RSS     AIC
## - sumsC2  1   0.23457 144.77 -1114.2
## - sumsO9  1   0.32229 144.86 -1113.7
## - sumsE8  1   0.34433 144.88 -1113.6
## <none>                144.54 -1113.3
## - sumsC1  1   0.43481 144.97 -1113.2
## - sumsC5  1   0.47911 145.02 -1112.9
## - sumsC3  1   0.48550 145.03 -1112.9
## - sumsO5  1   0.49006 145.03 -1112.9
## - sumsE1  1   0.49086 145.03 -1112.9
## - sumsE5  1   0.56698 145.11 -1112.5
## - sumsA3  1   0.67520 145.22 -1112.0
## - sumsC8  1   0.76363 145.30 -1111.5
## - sumsN5  1   0.80384 145.34 -1111.3
## - sumsC9  1   0.88702 145.43 -1110.9
## - sumsA7  1   0.89729 145.44 -1110.8
## - sumsO4  1   1.15272 145.69 -1109.6
## - sumsO1  1   1.15825 145.70 -1109.5
## - sumsA1  1   1.17427 145.71 -1109.5
## - sumsO6  1   1.25292 145.79 -1109.1
## - sumsO2  1   1.29065 145.83 -1108.9
## - sumsE2  1   1.68586 146.22 -1106.9
## - sumsN1  1   1.78787 146.33 -1106.4
## - sumsN3  1   1.82353 146.36 -1106.3
## - sumsO3  1   2.13752 146.68 -1104.7
## 
## Step:  AIC=-1114.15
## abs ~ sumsA1 + sumsA3 + sumsA7 + sumsC1 + sumsC3 + sumsC5 + sumsC8 + 
##     sumsC9 + sumsE1 + sumsE2 + sumsE5 + sumsE8 + sumsN1 + sumsN3 + 
##     sumsN5 + sumsO1 + sumsO2 + sumsO3 + sumsO4 + sumsO5 + sumsO6 + 
##     sumsO9
## 
##          Df Sum of Sq    RSS     AIC
## - sumsO9  1   0.28302 145.06 -1114.7
## - sumsC1  1   0.39523 145.17 -1114.2
## <none>                144.77 -1114.2
## - sumsE8  1   0.40922 145.18 -1114.1
## - sumsE1  1   0.44917 145.22 -1113.9
## - sumsO5  1   0.45014 145.22 -1113.9
## - sumsE5  1   0.53482 145.31 -1113.5
## - sumsC3  1   0.56209 145.34 -1113.3
## - sumsN5  1   0.64006 145.41 -1113.0
## - sumsC5  1   0.68351 145.46 -1112.8
## - sumsA3  1   0.71874 145.49 -1112.6
## - sumsC8  1   0.75618 145.53 -1112.4
## - sumsA7  1   0.93073 145.71 -1111.5
## - sumsC9  1   0.98067 145.75 -1111.3
## - sumsA1  1   1.17908 145.95 -1110.3
## - sumsO4  1   1.19784 145.97 -1110.2
## - sumsO1  1   1.20978 145.98 -1110.1
## - sumsO6  1   1.27297 146.05 -1109.8
## - sumsO2  1   1.28723 146.06 -1109.8
## - sumsN1  1   1.66039 146.44 -1107.9
## - sumsE2  1   1.87545 146.65 -1106.8
## - sumsN3  1   2.10980 146.88 -1105.7
## - sumsO3  1   2.18179 146.96 -1105.3
## 
## Step:  AIC=-1114.74
## abs ~ sumsA1 + sumsA3 + sumsA7 + sumsC1 + sumsC3 + sumsC5 + sumsC8 + 
##     sumsC9 + sumsE1 + sumsE2 + sumsE5 + sumsE8 + sumsN1 + sumsN3 + 
##     sumsN5 + sumsO1 + sumsO2 + sumsO3 + sumsO4 + sumsO5 + sumsO6
## 
##          Df Sum of Sq    RSS     AIC
## - sumsE8  1   0.37054 145.43 -1114.9
## - sumsC1  1   0.39338 145.45 -1114.8
## <none>                145.06 -1114.7
## - sumsE1  1   0.43259 145.49 -1114.6
## - sumsE5  1   0.47564 145.53 -1114.4
## - sumsO5  1   0.48499 145.54 -1114.3
## - sumsC3  1   0.52771 145.59 -1114.1
## - sumsC5  1   0.58193 145.64 -1113.8
## - sumsN5  1   0.59791 145.66 -1113.8
## - sumsA3  1   0.72638 145.78 -1113.1
## - sumsC9  1   0.75090 145.81 -1113.0
## - sumsC8  1   0.76286 145.82 -1113.0
## - sumsA7  1   0.87056 145.93 -1112.4
## - sumsA1  1   1.16757 146.22 -1111.0
## - sumsO1  1   1.18733 146.25 -1110.8
## - sumsO2  1   1.33507 146.39 -1110.1
## - sumsO4  1   1.33983 146.40 -1110.1
## - sumsO6  1   1.51340 146.57 -1109.2
## - sumsE2  1   1.80911 146.87 -1107.8
## - sumsN1  1   1.81947 146.88 -1107.7
## - sumsN3  1   2.03650 147.09 -1106.7
## - sumsO3  1   2.57741 147.63 -1104.0
## 
## Step:  AIC=-1114.89
## abs ~ sumsA1 + sumsA3 + sumsA7 + sumsC1 + sumsC3 + sumsC5 + sumsC8 + 
##     sumsC9 + sumsE1 + sumsE2 + sumsE5 + sumsN1 + sumsN3 + sumsN5 + 
##     sumsO1 + sumsO2 + sumsO3 + sumsO4 + sumsO5 + sumsO6
## 
##          Df Sum of Sq    RSS     AIC
## - sumsE5  1   0.38387 145.81 -1115.0
## <none>                145.43 -1114.9
## - sumsC1  1   0.41315 145.84 -1114.8
## - sumsO5  1   0.43536 145.86 -1114.7
## - sumsC5  1   0.52120 145.95 -1114.3
## - sumsN5  1   0.61353 146.04 -1113.8
## - sumsC3  1   0.68102 146.11 -1113.5
## - sumsC9  1   0.73215 146.16 -1113.3
## - sumsA3  1   0.76525 146.19 -1113.1
## - sumsC8  1   0.77734 146.21 -1113.0
## - sumsA7  1   0.83715 146.26 -1112.8
## - sumsE1  1   0.90537 146.33 -1112.4
## - sumsA1  1   0.98381 146.41 -1112.0
## - sumsO4  1   1.24421 146.67 -1110.7
## - sumsO1  1   1.30657 146.73 -1110.4
## - sumsO6  1   1.37523 146.80 -1110.1
## - sumsO2  1   1.81192 147.24 -1108.0
## - sumsE2  1   1.83674 147.26 -1107.8
## - sumsN1  1   1.89050 147.32 -1107.6
## - sumsN3  1   2.06036 147.49 -1106.7
## - sumsO3  1   2.81212 148.24 -1103.1
## 
## Step:  AIC=-1114.99
## abs ~ sumsA1 + sumsA3 + sumsA7 + sumsC1 + sumsC3 + sumsC5 + sumsC8 + 
##     sumsC9 + sumsE1 + sumsE2 + sumsN1 + sumsN3 + sumsN5 + sumsO1 + 
##     sumsO2 + sumsO3 + sumsO4 + sumsO5 + sumsO6
## 
##          Df Sum of Sq    RSS     AIC
## - sumsC1  1   0.23294 146.04 -1115.8
## <none>                145.81 -1115.0
## - sumsO5  1   0.40455 146.22 -1115.0
## - sumsC5  1   0.50585 146.32 -1114.5
## - sumsA3  1   0.57814 146.39 -1114.1
## - sumsN5  1   0.58220 146.39 -1114.1
## - sumsE1  1   0.64668 146.46 -1113.8
## - sumsC3  1   0.69382 146.50 -1113.6
## - sumsC8  1   0.81093 146.62 -1113.0
## - sumsC9  1   0.81554 146.63 -1113.0
## - sumsA7  1   1.02406 146.84 -1111.9
## - sumsA1  1   1.04857 146.86 -1111.8
## - sumsO6  1   1.22695 147.04 -1110.9
## - sumsO4  1   1.24638 147.06 -1110.8
## - sumsO1  1   1.60912 147.42 -1109.1
## - sumsO2  1   1.88121 147.69 -1107.7
## - sumsE2  1   1.93889 147.75 -1107.5
## - sumsN3  1   2.15778 147.97 -1106.4
## - sumsN1  1   2.20066 148.01 -1106.2
## - sumsO3  1   2.56485 148.38 -1104.4
## 
## Step:  AIC=-1115.84
## abs ~ sumsA1 + sumsA3 + sumsA7 + sumsC3 + sumsC5 + sumsC8 + sumsC9 + 
##     sumsE1 + sumsE2 + sumsN1 + sumsN3 + sumsN5 + sumsO1 + sumsO2 + 
##     sumsO3 + sumsO4 + sumsO5 + sumsO6
## 
##          Df Sum of Sq    RSS     AIC
## <none>                146.04 -1115.8
## - sumsO5  1   0.42409 146.47 -1115.7
## - sumsC5  1   0.43120 146.48 -1115.7
## - sumsN5  1   0.55806 146.60 -1115.1
## - sumsC3  1   0.60266 146.65 -1114.9
## - sumsE1  1   0.66757 146.71 -1114.5
## - sumsC8  1   0.72049 146.76 -1114.3
## - sumsC9  1   0.80044 146.84 -1113.9
## - sumsA7  1   0.86761 146.91 -1113.6
## - sumsA1  1   1.08217 147.13 -1112.5
## - sumsO4  1   1.27071 147.31 -1111.6
## - sumsA3  1   1.33390 147.38 -1111.3
## - sumsO6  1   1.39450 147.44 -1111.0
## - sumsO1  1   1.56323 147.61 -1110.2
## - sumsE2  1   1.77989 147.82 -1109.1
## - sumsO2  1   1.83557 147.88 -1108.8
## - sumsN1  1   2.02666 148.07 -1107.9
## - sumsN3  1   2.16103 148.21 -1107.2
## - sumsO3  1   2.57776 148.62 -1105.2
\end{verbatim}

\begin{itemize}
\tightlist
\item
  H3. Facets will improve the predictive power of dimensions when
  predicting school abseentism.
\end{itemize}

H3 will also be tested with a stepwise regression in which the Big Five
dimensions will be first entered and then the full set of facets.

Our first set of hypothesis tested how personality was related to SWL.
Extraversion (\emph{r} = 0.33) and Neuroticism (\emph{r} = 0.40) were
the dimensions with higher correlations with SWL. In H1.1, the model
which included the facets outperformed the dimension model (\emph{F} =
17.89, p \textless{} 0.001). The model resulted in a predictive gain of
\(\Delta R^2\) = 0.17. Adding N2 and E4 result in a predictive gain of
\(\Delta R^2\) = 0.12.

Our second set of hypothesis involves predictions to academic
achievement. Conscientiousness correlated with academic achievement with
\emph{r} = 0.24, being the strongest correlation of all the set of
dimensions. Openness correlated \emph{r} = 0.17 with the criterion. For
H2.2, the model which included the facets again outperformed the
dimensional model (\emph{F} = 2.07, p \textless{} 0.001), with a
predictive gain of \(\Delta R^2\) = 0.09.

Our thirst set of hypothesis explored the relationship of personality
with school absences. The facet level model outperformed the dimensional
level (\emph{F} = 6.80, p \textless{} 0.001), \(\Delta R^2\) = 0.11.

\vspace{5mm}

\textless{} Table 5 here caption=\enquote{Criterion correlations}
\textgreater{}

\vspace{5mm}

\subsection{Study 2 -- German Sample}\label{study-2-german-sample}

\subsubsection{Participants}\label{participants-1}

The representative sample consisted of 387 German speakers (49.10\%
male) with a mean age of 45.60 years (SD = 17.50). The data was
collected in a test center.

\subsubsection{Measures}\label{measures-1}

The five items per facet derived from Study 1 were translated and
back-translated by bilingual experts, creating a German version of the
measure used there. The translated items can be found in appendix B.

\subsection{Procedure}\label{procedure-1}

\subsubsection{Step 1 -- Examining the
structure.}\label{step-1-examining-the-structure.}

To check the facet structure Study 1 delivered, multiple confirmatory
factor analyses were calculated via Mplus following an analogue
procedure to Study 1. First, measurement models were estimated for all
facets, using WLSMV as the estimator. Model fit was determined based on
the guide lines mentioned above. In a final model, all five domain
structural models were integrated using ESEM.

\subsubsection{Step 2 -- Testing for measurement
invariance.}\label{step-2-testing-for-measurement-invariance.}

In a next step, measurement invariance between German and US samples was
examined. We followed the procedure suggested by Sass (2011) and tested
configural, factorial and strong factorial invariance. The cutoffs
suggested by Chen (2007) were applied to compare model fits. According
to this configural measurement invariance can be assumed when the same
item is associated with the same factor in each domain, while the factor
loadings can differ. If the factor loadings of each item would not
differ between the samples, factorial measurement invariance can be
assumed. Strong factorial measurement invariance can be assumed when on
top of that the intercepts of each item are equal. The limit to
factorial measurement invariance was set to \(\Delta\) CFI \textless{}
.01, \(\Delta\) RMSEA \textless{} .015 and \(\Delta\) SRMR \textless{}
.03, at which the limit to strong factorial measurement invariance was
set to \(\Delta\) CFI \textless{} .01, \(\Delta\) RMSEA \textless{}
.015, \(\Delta\) SRMR \textless{} .01 as suggested by Chen (2007).

\subsection{Results}\label{results-1}

\subsubsection{Results of CFA}\label{results-of-cfa}

The measurement models of the American sample were replicated for the
reduced number of items per facet. Model fits can also be seen in
\emph{Table 3}. The ESEM with all five domains showed a relatively good
fit to the data with CFI = .82, RMSEA = .078, SRMR = .044. \emph{Table
6} shows the ESEM factor loadings for the German sample. All facets
loaded significantly on their intended domain.

\vspace{5mm}

\vspace{5mm}

\subsubsection{Results of MI}\label{results-of-mi}

For analyzing measurement invariance the latest facet model structure
(with additional facets) was taken. The results are shown in Table 7.
Configural measurement invariance could be shown for the facets
\emph{Appreciation of others}, \emph{Superiority/Grandiosity},
\emph{Need to be liked}, Crybabiness, Manipulation, Altruism (facets of
Agreeableness), Perseverance, Task Planning,
Goal-orientation/Achievement striving, Preferred Load, Procrastination
(facets of Conscientiousness), Assertiveness,
Sociability/Gregariousness, Activity (facets of Extraversion),
Irritability, Self-serving Attention (facets of Neuroticism),
Self-attributed Inginuity, Openness to actions and activities,
Openmindedness/Judgement, Love of Learning, Openness to feelings and
Intellect (facets of Openness).

Factorial measurement invariance could be shown for the facets Meanness,
Trust (facets of Agreeableness), Control of others, Lack of (Self-)
Control, Deliberation/Caution, Lack of Tidiness/Order (facets of
Conscientiousness), Sensation Seeking, Reclusiveness, Emotionality,
Humor (facets of Extraversion), Depression, Anxiety, Self-assuredness,
Lethargia, Sentimentality (facets of Neuroticism), Openness to reading,
Openness to arts and Need for cognition (facets of Openness).

The only facet with strong factorial measurement invariance was Shyness,
a facet of Extraversion

\vspace{5mm}

\vspace{5mm}

\section{Discussion}\label{discussion}

We have presented in this work an open-access instrument for personality
assessment within the Big Five framework, which showed evidences of
factorial validity in two different cultures and maximized the space set
of facets encompassed. With a modest number of items (202) by comparison
with the most influential Big Five inventories presented in \emph{Table
1}, we have reached to a large set of facets which mostly show a robust
factorial validity in both studies, as shown in \emph{Table 3}.

The Big Five solution has been recognized as the most replicable model
for personality inventories, reaching a hallmark of consensus in
personality science for the last decades. However, some researchers have
pointed out that while the Big Five has repeatedely been found when
fitting EFA to personality data, its replicability under CFA procedures
has been more elusive (R. R. McCrae, Zonderman, Costa, Bond, \&
Paunonen, 1996). The constriction of the common independent cluster
solution, where cross-loadings are restricted to zero, may suppose a
rather strong assumption for personality trait inventories (Marsh et
al., 2010). The idea of facets, or habits, being influenced by more than
one domain can definitely make some sense. ESEM helps overcoming this
assumption and provides a measure about how well the Big Five solution
adjusts to the data. Using this procedure, the degree of integration of
our proposed set of facets to the Big Five factor solution has been
solid enough according to the cut-off values proposed by Marsh et al.
(2010). The number of significant cross-loadings in the ESEM models has
not been large either, advocating a good discriminant validity.

The instrument presented in this work covers all the \enquote{core}
facets proposed by Christopher J. Soto and John (2009), either directly
or indirectly. The \emph{Energy} construct in Extraversion is literally
covered by a three-item facet in our instrument, whereas the
\emph{Assertiveness} construct has been tapped by items belonging to the
\emph{Wish for affiliation}, \emph{Communicativeness} and
\emph{Conviviality} facets. \emph{Altruism} is directly reflected in a
five-item facet, while the \emph{Compliance} construct is reflected by
our \emph{Good faith} facet. The \emph{Order} and \emph{Self-discipline}
constructs proposed by Christopher J. Soto and John (2009) are mirrored
by dedicated facets in our instrument. The \emph{Anxiety} and
\emph{Depression} constructs are mirrored by the facets \emph{Mental
balance} and \emph{Emotional robustness}, respectively. For the
\emph{Openess} dimension, the \emph{Aesthetic} contruct is covered by
our facet \emph{Artistic interest}, while the \emph{Ideas} construct has
been reflected by both the \emph{Open-mindedness} and the \emph{Wish to
analyze} facets. The two-per-facet components proposed by DeYoung et al.
(2007) were also being tapped by the set of facets in our inventory.

The instrument covers most of facets proposed by the most influential
Big Five measures as seen in \emph{Table 1}. The most salient
differences are related to the HEXACO model, which entails a six factor
solution with a slightly different theoretical conceptualization (K. Lee
\& Ashton, 2006). Most notably

Although these facets are not being covered directly in our inventory,
components of facets from distinct domains in our model retain a glimpse
of the missing facets. This underlies the importance of allowing
cross-loadings for trait personality data. Let's use the example of
\emph{Patience}, a facet proposed in the HEXACO model for the
Agreeableness domain which is not covered in our instrument, nor in the
other three Big Five inventories which have been revised. Although
patience, there is a notion of a patient trait within the
Self-discipline facet in the Conscientiousness domain, specially with
items such as \enquote{I rush into things} or \enquote{I act impulsively
when something is bothering me} (See appendix A). In fact,
Self-discipline has important cross-loadings with Agreeableness in both
samples (\(\lambda\) = .256 in the USA sample and \(\lambda\) = .341 in
the german sample).

In addition we included even more facets.

In addition, evidences for external criteria validity were attained.

We have collected some criterion validity evidences. Like bla bla bla.
Nonetheless the multi - facetted nature of the instrument makes
forthcoming evidences for criterion and predictive validity promising.

One limitation is the sample used. Students are not a representative
population of society and results may not be generalized.

Future directions are to provide a tool with the subset of items for
public use. Gather community sample, from more cultures and test the
extent of the universality of the instrument. And use the instrument to
predict important life outcomes so the links between specific behaviors
and facets become richer.

\newpage

Remove this page. This is used to include the tables' references into
the bibliography.

Brick and Lewis (2014); Gaughan, Miller, and Lynam (2012); Leone,
Chirumbolo, and Desimoni (2012); Mcabee, Oswald, and Connelly (2014);
Gaughan, Miller, Pryor, and Lynam (2009); Noftle and Shaver (2006); R.
M. Bagby, Taylor, and Parker (1994); Schimmack, Furr, and Funder (1999);
Wakabayashi, Baron-Cohen, and Wheelwright (2006); Shaver and Brennan
(1992); Ruiz, Pincus, and Dickinson (2003); Mccrae, Kurtz, Yamagata, and
Terracciano (2011); Rosander, Bäckström, and Stenberg (2011); K. K.
McAdams and Donnellan (2009); Siddiqui (2011); Hagger-Johnson and
Whiteman (2007)

Ziegler et al. (2014)

\newpage

\section{References}\label{references}

\begingroup
\setlength{\parindent}{-0.5in} \setlength{\leftskip}{0.5in}

\hypertarget{refs}{}
\hypertarget{ref-AllportOdbert1936}{}
Allport, G. W., \& Odbert, H. S. (1936). Trait-names: A psycho-lexical
study. \emph{Psychological Monographs}, \emph{47}(1), i--171.
doi:\href{https://doi.org/10.1037/h0093360}{10.1037/h0093360}

\hypertarget{ref-APA2013}{}
American Psychiatric Association. (2013). \emph{Diagnostic and
statistical manual of mental disorders (5th ed.)}.

\hypertarget{ref-AsparouhovMuthen2009}{}
Asparouhov, T., \& Muthén, B. (2009). \emph{Exploratory structural
equation modeling} (Vol. 16, pp. 397--438).
doi:\href{https://doi.org/10.1080/10705510903008204}{10.1080/10705510903008204}

\hypertarget{ref-Bagby2018}{}
Bagby, R. M., \& Widiger, T. A. (2018). Five factor model personality
disorder scales: An introduction to a special section on assessment of
maladaptive variants of the five factor model. \emph{Psychological
Assessment}, \emph{30}(1), 1--9.
doi:\href{https://doi.org/10.1037/pas0000523}{10.1037/pas0000523}

\hypertarget{ref-Bagby1994}{}
Bagby, R. M., Taylor, G. J., \& Parker, J. D. (1994). The twenty-item
Toronto Alexithymia scale-II. Convergent, discriminant, and concurrent
validity. \emph{Journal of Psychosomatic Research}, \emph{38}(1),
33--40.
doi:\href{https://doi.org/10.1016/0022-3999(94)90006-X}{10.1016/0022-3999(94)90006-X}

\hypertarget{ref-Beauducel2005}{}
Beauducel, A., \& Wittmann, W. (2005). Simulation study on fit indices
in confirmatory factor analyses based on data with slightly distorted
simple structure. \emph{Structural Equation Modeling}, \emph{12},
41--75.
doi:\href{https://doi.org/10.1207/s15328007sem1201}{10.1207/s15328007sem1201}

\hypertarget{ref-Borgatta1964}{}
Borgatta, E. (1964). The structure of personality characteristics.
\emph{Behavioral Science}, \emph{9}(1), 8--17.
doi:\href{https://doi.org/10.1007/BF01358190}{10.1007/BF01358190}

\hypertarget{ref-Brick2014}{}
Brick, C., \& Lewis, G. J. (2014). Unearthing the ``Green'' Personality:
Core Traits Predict Environmentally Friendly Behavior. \emph{Environment
and Behavior}, \emph{48}(5), 635--658.
doi:\href{https://doi.org/10.1177/0013916514554695}{10.1177/0013916514554695}

\hypertarget{ref-Cattell1956}{}
Cattell, R. B. (1956). Second-order personality factors in the
questionnaire realm. \emph{Journal of Consulting Psychology},
\emph{20}(6), 411--418.
doi:\href{https://doi.org/10.1037/h0047239}{10.1037/h0047239}

\hypertarget{ref-Chamorro-Premuzic2003}{}
Chamorro-Premuzic, T., \& Furnham, A. (2003). Personality predicts
academic performance: Evidence from two longitudinal university samples.
doi:\href{https://doi.org/10.1016/S0092-6566(02)00578-0}{10.1016/S0092-6566(02)00578-0}

\hypertarget{ref-Chen2007}{}
Chen, F. F. (2007). Sensitivity of goodness of fit indexes to lack of
measurement invariance. \emph{Structural Equation Modeling},
\emph{14}(3), 464--504.
doi:\href{https://doi.org/10.1080/10705510701301834}{10.1080/10705510701301834}

\hypertarget{ref-Clark2005}{}
Clark, L. A. (2005). Temperament as a unifying basis for personality and
psychopathology. \emph{Journal of Abnormal Psychology}, \emph{114}(4),
505--521.
doi:\href{https://doi.org/10.1037/0021-843X.114.4.505}{10.1037/0021-843X.114.4.505}

\hypertarget{ref-Widiger1994}{}
Costa Jr., P. T., \& Widiger, T. A. (1994). A description of the
DSM-III-R and DSM-IV personality disorders with the five-factor model of
personality. \emph{Personality Disorders and the Five-Factor Model of
Personality.}, (January), 41--56.
doi:\href{https://doi.org/10.1037/10140-003}{10.1037/10140-003}

\hypertarget{ref-Costa1995}{}
Costa, P. T., \& McCrae, R. R. (1995). Domains and facets: hierarchical
personality assessment using the revised NEO personality inventory.
\emph{Journal of Personality Assessment}, \emph{64}(1), 21--50.
doi:\href{https://doi.org/10.1207/s15327752jpa6401_2}{10.1207/s15327752jpa6401\_2}

\hypertarget{ref-DeFruyt1996}{}
De Fruyt, F., \& Mervielde, I. (1996). Personality and interests as
predictors of educational streaming and achievement. \emph{European
Journal of Personality}, \emph{10}(5), 405--425.
doi:\href{https://doi.org/10.1002/(SICI)1099-0984(199612)10:5\%3C405::AID-PER255\%3E3.0.CO;2-M}{10.1002/(SICI)1099-0984(199612)10:5\textless{}405::AID-PER255\textgreater{}3.0.CO;2-M}

\hypertarget{ref-DeYoung2007}{}
DeYoung, C. G., Quilty, L. C., \& Peterson, J. B. (2007). Between Facets
and Domains: 10 Aspects of the Big Five. \emph{Journal of Personality
and Social Psychology}, \emph{93}(5), 880--896.
doi:\href{https://doi.org/10.1037/0022-3514.93.5.880}{10.1037/0022-3514.93.5.880}

\hypertarget{ref-Diener1985}{}
Diener, E., Emmons, R. A., Larsen, R. J., \& Griffin, S. (1985). The
Satisfaction With Life Scale. \emph{Journal of Personality},
\emph{49}(1), 71--75.
doi:\href{https://doi.org/10.1207/s15327752jpa4901}{10.1207/s15327752jpa4901}

\hypertarget{ref-Diener2003}{}
Diener, E., Oishi, S., \& Lucas, R. E. (2003). Personality, culture, and
subjective well-being.
doi:\href{https://doi.org/10.1146/annurev.psych.54.101601.145056}{10.1146/annurev.psych.54.101601.145056}

\hypertarget{ref-Digman1990}{}
Digman, J. M. (1990). Personality Structure: Emergence of the
Five-Factor Model. \emph{Annual Review of Psychology}, \emph{41}(1),
417--440.
doi:\href{https://doi.org/10.1146/annurev.ps.41.020190.002221}{10.1146/annurev.ps.41.020190.002221}

\hypertarget{ref-Fiske1949}{}
Fiske, D. W. (1949). Consistency of the factorial structures of
personality ratings from different sources. \emph{Journal of Abnormal
and Social Psychology}, \emph{44}(3), 329--344.
doi:\href{https://doi.org/10.1037/h0057198}{10.1037/h0057198}

\hypertarget{ref-Galton1884}{}
Galton, F. (1884). The Measurement of Character.
doi:\href{https://doi.org/10.1037/11352-058}{10.1037/11352-058}

\hypertarget{ref-Gaughan2012}{}
Gaughan, E. T., Miller, J. D., \& Lynam, D. R. (2012). Examining the
Utility of General Models of Personality in the Study of Psychopathy: A
Comparison of the HEXACO-PI-R and NEO PI-R. \emph{Journal of Personality
Disorders}, \emph{26}(4), 513--523.
doi:\href{https://doi.org/10.1521/pedi.2012.26.4.513}{10.1521/pedi.2012.26.4.513}

\hypertarget{ref-Gaughan2009}{}
Gaughan, E. T., Miller, J. D., Pryor, L. R., \& Lynam, D. R. (2009).
Comparing two alternative measures of general personality in the
assessment of psychopathy: A test of the NEO PI-R and the MPQ.
\emph{Journal of Personality}, \emph{77}(4), 965--995.
doi:\href{https://doi.org/10.1111/j.1467-6494.2009.00571.x}{10.1111/j.1467-6494.2009.00571.x}

\hypertarget{ref-Goldberg2006}{}
Goldberg, L. R., Johnson, J. A., Eber, H. W., Hogan, R., Ashton, M. C.,
Cloninger, C. R., \& Gough, H. G. (2006). The international personality
item pool and the future of public-domain personality measures.
\emph{Journal of Research in Personality}, \emph{40}(1), 84--96.
doi:\href{https://doi.org/10.1016/j.jrp.2005.08.007}{10.1016/j.jrp.2005.08.007}

\hypertarget{ref-Hagger-Johnson2007}{}
Hagger-Johnson, G. E., \& Whiteman, M. C. (2007). Conscientiousness
facets and health behaviors: A latent variable modeling approach.
\emph{Personality and Individual Differences}, \emph{43}(5), 1235--1245.
doi:\href{https://doi.org/10.1016/j.paid.2007.03.014}{10.1016/j.paid.2007.03.014}

\hypertarget{ref-Horn1965}{}
Horn, J. L. (1965). A rationale and test for the number of factors in
factor analysis. \emph{Psychometrika}, \emph{30}(2), 179--185.
doi:\href{https://doi.org/10.1007/BF02289447}{10.1007/BF02289447}

\hypertarget{ref-Hu1999}{}
Hu, L. T., \& Bentler, P. M. (1999). Cutoff criteria for fit indexes in
covariance structure analysis: Conventional criteria versus new
alternatives. \emph{Structural Equation Modeling}, \emph{6}(1), 1--55.
doi:\href{https://doi.org/10.1080/10705519909540118}{10.1080/10705519909540118}

\hypertarget{ref-Judge1997}{}
Judge, T. A., Martocchio, J. J., \& Thoresen, C. J. (1997). Five-Factor
Model of Personality and Employee Absense. \emph{Journal of Applied
Psychology}, \emph{82}(5), 11. Retrieved from
\href{c:\%7B/\%\%7D5CDocuments\%20and\%20Settings\%7B/\%\%7D5Ce8902872\%7B/\%\%7D5CDesktop\%7B/\%\%7D5Cdata\%20disk\%7B/\%\%7D5CLibrary\%7B/\%\%7D5CCURRENT\%7B/\%\%7D5CEndNote\%7B/\%\%7D5CCATALOGUED\%20+\%20LINKED\%7B/\%\%7D5CJudgeetal1997.pdf}{c:\{\textbackslash{}\%\}5CDocuments and Settings\{\textbackslash{}\%\}5Ce8902872\{\textbackslash{}\%\}5CDesktop\{\textbackslash{}\%\}5Cdata disk\{\textbackslash{}\%\}5CLibrary\{\textbackslash{}\%\}5CCURRENT\{\textbackslash{}\%\}5CEndNote\{\textbackslash{}\%\}5CCATALOGUED + LINKED\{\textbackslash{}\%\}5CJudgeetal1997.pdf}

\hypertarget{ref-Krueger2012a}{}
Krueger, R. F., Derringer, J., Markon, K. E., Watson, D., \& Skodol, A.
E. (2012). Initial construction of a maladaptive personality trait model
and inventory for DSM ­ 5 Initial construction of a maladaptive
personality trait model and inventory for DSM-5. \emph{Psychological
Medicine}, \emph{42}(09), 1872--1890.
doi:\href{https://doi.org/10.1017/S0033291711002674}{10.1017/S0033291711002674}

\hypertarget{ref-Lee2006}{}
Lee, K., \& Ashton, M. C. (2006). Further assessment of the HEXACO
personality inventory: Two new facet scales and an observer report form.
\emph{Psychological Assessment}, \emph{18}(2), 182--191.
doi:\href{https://doi.org/10.1037/1040-3590.18.2.182}{10.1037/1040-3590.18.2.182}

\hypertarget{ref-Lee2016}{}
Lee, K., \& Ashton, M. C. (2016). Psychometric Properties of the
HEXACO-100. \emph{Assessment}, \emph{1-15}.
doi:\href{https://doi.org/10.1177/1073191116659134}{10.1177/1073191116659134}

\hypertarget{ref-Leone2012}{}
Leone, L., Chirumbolo, A., \& Desimoni, M. (2012). The impact of the
HEXACO personality model in predicting socio-political attitudes: The
moderating role of interest in politics. \emph{Personality and
Individual Differences}, \emph{52}(3), 416--421.
doi:\href{https://doi.org/10.1016/j.paid.2011.10.049}{10.1016/j.paid.2011.10.049}

\hypertarget{ref-Lievens2002}{}
Lievens, F., Coetsier, P., De Fruyt, F., \& De Maeseneer, J. (2002).
Medical students' personality characteristics and academic performance:
A five-factor model perspective. \emph{Medical Education},
\emph{36}(11), 1050--1056.
doi:\href{https://doi.org/10.1046/j.1365-2923.2002.01328.x}{10.1046/j.1365-2923.2002.01328.x}

\hypertarget{ref-Lounsbury2004}{}
Lounsbury, J. W., Steel, R. P., Loveland, J. M., \& Gibson, L. W.
(2004). An investigation of personality traits in relation to adolescent
school absenteeism. \emph{Journal of Youth and Adolescence},
\emph{33}(5), 457--466.
doi:\href{https://doi.org/10.1023/B:JOYO.0000037637.20329.97}{10.1023/B:JOYO.0000037637.20329.97}

\hypertarget{ref-Lounsbury2002}{}
Lounsbury, J. W., Sundstrom, E., Loveland, J. L., \& Gibson, L. W.
(2002). Broad versus narrow personality traits in predicting academic
performance of adolescents. \emph{Learning and Individual Differences},
\emph{14}(1), 67--77.
doi:\href{https://doi.org/10.1016/j.lindif.2003.08.001}{10.1016/j.lindif.2003.08.001}

\hypertarget{ref-MacCann2009}{}
MacCann, C., Duckworth, A. L., \& Roberts, R. D. (2009). Empirical
identification of the major facets of Conscientiousness. \emph{Learning
and Individual Differences}, \emph{19}(4), 451--458.
doi:\href{https://doi.org/10.1016/j.lindif.2009.03.007}{10.1016/j.lindif.2009.03.007}

\hypertarget{ref-Markon2013}{}
Markon, K. E., Quilty, L. C., Bagby, R. M., \& Krueger, R. F. (2013).
The Development and Psychometric Properties of an Informant-Report Form
of the Personality Inventory for DSM-5 (PID-5). \emph{Assessment},
\emph{20}(3), 370--383.
doi:\href{https://doi.org/10.1177/1073191113486513}{10.1177/1073191113486513}

\hypertarget{ref-Marsh2010}{}
Marsh, H. W., Lüdtke, O., Muthén, B., Asparouhov, T., Morin, A. J.,
Trautwein, U., \& Nagengast, B. (2010). A New Look at the Big Five
Factor Structure Through Exploratory Structural Equation Modeling.
\emph{Psychological Assessment}, \emph{22}(3), 471--491.
doi:\href{https://doi.org/10.1037/a0019227}{10.1037/a0019227}

\hypertarget{ref-Mcabee2014}{}
Mcabee, S. T., Oswald, F. L., \& Connelly, B. S. (2014). Bifactor Models
of Personality and College Student Performance: A Broad Versus Narrow
View. \emph{European Journal of Personality}, \emph{28}(6), 604--619.
doi:\href{https://doi.org/10.1002/per.1975}{10.1002/per.1975}

\hypertarget{ref-McAdams2006a}{}
McAdams, D. P., \& Pals, J. L. (2006). A new Big Five: Fundamental
principles for an integrative science of personality. \emph{American
Psychologist}, \emph{61}(3), 204--217.
doi:\href{https://doi.org/10.1037/0003-066X.61.3.204}{10.1037/0003-066X.61.3.204}

\hypertarget{ref-McAdams2009}{}
McAdams, K. K., \& Donnellan, M. B. (2009). Facets of personality and
drinking in first-year college students. \emph{Personality and
Individual Differences}, \emph{46}(2), 207--212.
doi:\href{https://doi.org/10.1016/j.paid.2008.09.028}{10.1016/j.paid.2008.09.028}

\hypertarget{ref-Mccrae2011}{}
Mccrae, R. R., Kurtz, J. E., Yamagata, S., \& Terracciano, A. (2011).
Internal consistency, retest reliability and their implications for
personality Scale Validity. \emph{Personality and Social Psychological
Bulletin}, \emph{15}(1), 28--50.
doi:\href{https://doi.org/10.1177/1088868310366253.Internal}{10.1177/1088868310366253.Internal}

\hypertarget{ref-McCrae1996}{}
McCrae, R. R., Zonderman, A. B., Costa, P. T., Bond, M. H., \& Paunonen,
S. V. (1996). Evaluating replicability of factors in the tevised NEO
personality inventory: Confirmatory factor analysis versus procrustes
rotation. \emph{Journal of Personality and Social Psychology},
\emph{70}(3), 552--566. Retrieved from
\url{http://www.sciencedirect.com/science/article/B6X01-46SGF6X-B/2/cfbcc79b23f57818759b3ae2b7f949b5}

\hypertarget{ref-Noftle2007}{}
Noftle, E. E., \& Robins, R. W. (2007). Personality Predictors of
Academic Outcomes: Big Five Correlates of GPA and SAT Scores.
\emph{Journal of Personality and Social Psychology}, \emph{93}(1),
116--130.
doi:\href{https://doi.org/10.1037/0022-3514.93.1.116}{10.1037/0022-3514.93.1.116}

\hypertarget{ref-Noftle2006}{}
Noftle, E. E., \& Shaver, P. R. (2006). Attachment dimensions and the
big five personality traits: Associations and comparative ability to
predict relationship quality. \emph{Journal of Research in Personality},
\emph{40}(2), 179--208.
doi:\href{https://doi.org/10.1016/j.jrp.2004.11.003}{10.1016/j.jrp.2004.11.003}

\hypertarget{ref-Norman1967}{}
Norman, W. T. (1967). 2800 Personality Trait Descriptors - Normative
Operating Characteristics for a University Population, 1--279.

\hypertarget{ref-Ones2003}{}
Ones, D. S., Viswesvaran, C., \& Schmidt, F. L. (2003). Personality and
absenteeism: a meta analysis of integrity tests. \emph{European Journal
of Personality}, \emph{17}(S1), S19--S38.
doi:\href{https://doi.org/10.1002/per.487}{10.1002/per.487}

\hypertarget{ref-OzerBenet2006}{}
Ozer, D. J., \& Benet-Martínez, V. (2006). Personality and the
Prediction of Consequential Outcomes. \emph{Annual Review of
Psychology}, \emph{57}(1), 401--421.
doi:\href{https://doi.org/10.1146/annurev.psych.57.102904.190127}{10.1146/annurev.psych.57.102904.190127}

\hypertarget{ref-OConnor2007}{}
O'Connor, M. C., \& Paunonen, S. V. (2007). Big Five personality
predictors of post-secondary academic performance. \emph{Personality and
Individual Differences}, \emph{43}(5), 971--990.
doi:\href{https://doi.org/10.1016/j.paid.2007.03.017}{10.1016/j.paid.2007.03.017}

\hypertarget{ref-Paunonen2001}{}
Paunonen, S. V., \& Ashton, M. C. (2001). Big Five Predictors of
Academic Achievement. \emph{Journal of Research in Personality},
\emph{35}(1), 78--90.
doi:\href{https://doi.org/10.1006/jrpe.2000.2309}{10.1006/jrpe.2000.2309}

\hypertarget{ref-Poropat2009}{}
Poropat, A. E. (2009). A Meta-Analysis of the Five-Factor Model of
Personality and Academic Performance. \emph{Psychological Bulletin},
\emph{135}(2), 322--338.
doi:\href{https://doi.org/10.1037/a0014996}{10.1037/a0014996}

\hypertarget{ref-Poropat2014}{}
Poropat, A. E. (2014). A meta-analysis of adult-rated child personality
and academic performance in primary education. \emph{British Journal of
Educational Psychology}, \emph{84}(2), 239--252.
doi:\href{https://doi.org/10.1111/bjep.12019}{10.1111/bjep.12019}

\hypertarget{ref-ClarkReynolds2001}{}
Reynolds, S. K., \& Clark, L. A. (2001). Predicting dimensions of
personality disorder from domains and facets of the Five-Factor Model.
\emph{Journal of Personality}, \emph{69}(2), 199--222.
doi:\href{https://doi.org/10.1111/1467-6494.00142}{10.1111/1467-6494.00142}

\hypertarget{ref-Roberts2007}{}
Roberts, B. W., Kuncel, N. R., Shiner, R., Caspi, A., \& Goldberg, L. R.
(2007). The Power of Personality: The Comparative Validity of
Personality Traits, Socioeconomic Status, and Cognitive Ability for
Predicting Important Life Outcomes. \emph{Perspectives on Psychological
Science}, \emph{2}(4), 313--345.
doi:\href{https://doi.org/10.1111/j.1745-6916.2007.00047.x}{10.1111/j.1745-6916.2007.00047.x}

\hypertarget{ref-Rosander2011}{}
Rosander, P., Bäckström, M., \& Stenberg, G. (2011). Personality traits
and general intelligence as predictors of academic performance: A
structural equation modelling approach. \emph{Learning and Individual
Differences}, \emph{21}(5), 590--596.
doi:\href{https://doi.org/10.1016/j.lindif.2011.04.004}{10.1016/j.lindif.2011.04.004}

\hypertarget{ref-Ruiz2003}{}
Ruiz, M. A., Pincus, A. L., \& Dickinson, K. A. (2003). NEO PI-R
predictors of alcohol use and alcohol-related problems. \emph{Journal of
Personality Assessment}, \emph{81}(3), 265--270.
doi:\href{https://doi.org/10.1207/S15327752JPA8103}{10.1207/S15327752JPA8103}

\hypertarget{ref-Salgado2002}{}
Salgado, J. F. (2002). The Big Five Personality Dimensions and
Counterproductive Behaviors. \emph{International Journal of Selection
and Assessment}, \emph{10}(1\&2), 117--125.
doi:\href{https://doi.org/10.1111/1468-2389.00198}{10.1111/1468-2389.00198}

\hypertarget{ref-SamuelWidiger2008}{}
Samuel, D. B., \& Widiger, T. A. (2008). A meta-analytic review of the
relationships between the five-factor model and DSM-IV-TR personality
disorders: A facet level analysis. \emph{Clinical Psychology Review},
\emph{28}(8), 1326--1342.
doi:\href{https://doi.org/10.1016/j.cpr.2008.07.002}{10.1016/j.cpr.2008.07.002}

\hypertarget{ref-Sass2011}{}
Sass, D. A. (2011). Testing measurement invariance and comparing latent
factor means within a confirmatory factor analysis framework.
\emph{Journal of Psychoeducational Assessment}, \emph{29}(4), 347--363.
doi:\href{https://doi.org/10.1177/0734282911406661}{10.1177/0734282911406661}

\hypertarget{ref-SaulsmanPage2004}{}
Saulsman, L. M., \& Page, A. C. (2004). The five-factor model and
personality disorder empirical literature: A meta-analytic review.
\emph{Clinical Psychology Review}, \emph{23}(8), 1055--1085.
doi:\href{https://doi.org/10.1016/j.cpr.2002.09.001}{10.1016/j.cpr.2002.09.001}

\hypertarget{ref-Schimmack2002}{}
Schimmack, U., Diener, E., \& Oishi, S. (2002). Life-satisfaction is a
momentary judgment and a stable personality characteristic: The use of
chronically accessible and stable sources. \emph{Journal of
Personality}, \emph{70}(3), 345--384.
doi:\href{https://doi.org/10.1111/1467-6494.05008}{10.1111/1467-6494.05008}

\hypertarget{ref-Schimmack1999}{}
Schimmack, U., Furr, R. M., \& Funder, D. C. (1999). Personality and
Life Satisfaction : A Facet-Level Analysis, 1062--1075.
doi:\href{https://doi.org/10.1177/0146167204264292}{10.1177/0146167204264292}

\hypertarget{ref-Schimmack2004}{}
Schimmack, U., Oishi, S., Furr, R. M., \& Funder, D. C. (2004).
Personality and life satisfaction: A facet-level analysis.
\emph{Personality and Social Psychology Bulletin}, \emph{30}(8),
1062--1075.
doi:\href{https://doi.org/10.1177/0146167204264292}{10.1177/0146167204264292}

\hypertarget{ref-Schmitt2007}{}
Schmitt, D. P., Allik, J., McCrae, R. R., Benet-Martínez, V., Alcalay,
L., Ault, L., \ldots{} Sharan, M. B. (2007). The geographic distribution
of Big Five personality traits: Patterns and profiles of human
self-description across 56 nations. \emph{Journal of Cross-Cultural
Psychology}.
doi:\href{https://doi.org/10.1177/0022022106297299}{10.1177/0022022106297299}

\hypertarget{ref-Seeboth2018}{}
Seeboth, A., \& Mõttus, R. (2018). Successful explanations start with
accurate descriptions: Questionnaire items as personality markers for
more accurate prediction and mapping of life outcomes. \emph{Journal of
Personality}.
doi:\href{https://doi.org/10.17605/OSF.IO/U65GB}{10.17605/OSF.IO/U65GB}

\hypertarget{ref-Shaver1992}{}
Shaver, P. R., \& Brennan, K. A. (1992). Attachment Styles and the ``Big
Five'' Personality Traits: Their Connections With Each Other and With
Romantic Relationship Outcomes. Society for Personality; Social
Psychology.

\hypertarget{ref-Siddiqui2011}{}
Siddiqui, K. (2011). Personality influences Mobile Phone usage.
\emph{Interdisciplinary Journal of \ldots{}}, (1981), 554--563.
Retrieved from
\href{http://papers.ssrn.com/abstract=2468985\%7B/\%\%7D0Ahttp://scholar.google.com/scholar?hl=en\%7B/\&\%7DbtnG=Search\%7B/\&\%7Dq=intitle:Personality+Influences+Mobile+Phone+Usage\%7B/\#\%7D4}{http://papers.ssrn.com/abstract=2468985\{\textbackslash{}\%\}0Ahttp://scholar.google.com/scholar?hl=en\{\textbackslash{}\&\}btnG=Search\{\textbackslash{}\&\}q=intitle:Personality+Influences+Mobile+Phone+Usage\{\textbackslash{}\#\}4}

\hypertarget{ref-SotoJohn2009}{}
Soto, C. J., \& John, O. P. (2009). Ten facet scales for the Big Five
Inventory: Convergence with NEO PI-R facets, self-peer agreement, and
discriminant validity. \emph{Journal of Research in Personality},
\emph{43}(1), 84--90.
doi:\href{https://doi.org/10.1016/j.jrp.2008.10.002}{10.1016/j.jrp.2008.10.002}

\hypertarget{ref-SotoJohn2016}{}
Soto, C. J., \& John, O. P. (2016). The Next Big Five Inventory ( BFI-2
): Developing and Assessing a Hierarchical Model With 15 Facets to
Enhance Bandwidth ... The Next Big Five Inventory ( BFI-2 ): Developing
and Assessing a Hierarchical Model With 15 Facets to Enhance Bandwidth ,
Fidelit, \emph{113}(June), 117--143.
doi:\href{https://doi.org/10.1037/pspp0000096}{10.1037/pspp0000096}

\hypertarget{ref-TupesChristal1961}{}
Tupes, E. C., \& Christal, R. E. (1961). Recurrent personality factors
based on trait rating. \emph{Lackland Air Force Base}, \emph{TX: USAF}.
Retrieved from
\href{https://ejwl.idm.oclc.org/login?url=http://search.ebscohost.com/login.aspx?direct=true\%7B/\&\%7Ddb=sih\%7B/\&\%7DAN=9208170745\%7B/\&\%7Dsite=ehost-live}{https://ejwl.idm.oclc.org/login?url=http://search.ebscohost.com/login.aspx?direct=true\{\textbackslash{}\&\}db=sih\{\textbackslash{}\&\}AN=9208170745\{\textbackslash{}\&\}site=ehost-live}

\hypertarget{ref-Velicer1976}{}
Velicer, W. F. (1976). Determining the number of components from the
matrix of partial correlations. \emph{Psychometrika}, \emph{41}(3).

\hypertarget{ref-Wakabayashi2006}{}
Wakabayashi, A., Baron-Cohen, S., \& Wheelwright, S. (2006). Are
autistic traits an independent personality dimension? A study of the
Autism-Spectrum Quotient (AQ) and the NEO-PI-R. \emph{Personality and
Individual Differences}, \emph{41}(5), 873--883.
doi:\href{https://doi.org/10.1016/j.paid.2006.04.003}{10.1016/j.paid.2006.04.003}

\hypertarget{ref-Watson2002}{}
Watson, D., \& Watson, D. (2002). General and Specific Traits of
Personality and Their Relation to Sleep and Academic Performance.
\emph{Journal of Personality}, \emph{70}(2), 177--206.
doi:\href{https://doi.org/10.1111/1467-6494.05002}{10.1111/1467-6494.05002}

\hypertarget{ref-WidigerMullins2009}{}
Widiger, T. A., \& Mullins-Sweatt, S. N. (2009). Five-Factor Model of
Personality Disorder: A Proposal for DSM-V. \emph{Annual Review of
Clinical Psychology}, \emph{5}(1), 197--220.
doi:\href{https://doi.org/10.1146/annurev.clinpsy.032408.153542}{10.1146/annurev.clinpsy.032408.153542}

\hypertarget{ref-Ziegler2014}{}
Ziegler, M., Bensch, D., Maaß, U., Schult, V., Vogel, M., \& Bühner, M.
(2014). Big Five facets as predictor of job training performance: The
role of specific job demands. \emph{Learning and Individual
Differences}, \emph{29}, 1--7.
doi:\href{https://doi.org/10.1016/j.lindif.2013.10.008}{10.1016/j.lindif.2013.10.008}

\hypertarget{ref-Ziegler2010}{}
Ziegler, M., Danay, E., Schölmerich, F., \& Bühner, M. (2010).
Predicting Academic Success with the Big 5 Rated from Different Points
of View: Self-Rated, Other Rated and Faked. \emph{European Journal of
Personality}, \emph{24}(July 2010), 341--355.
doi:\href{https://doi.org/10.1002/per}{10.1002/per}

\endgroup

\clearpage

\renewcommand{\listtablename}{Table captions}

\listoftables


\end{document}
